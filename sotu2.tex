\documentclass[a4paper, 11pt]{jsreport}
\usepackage{amsmath}
\usepackage{amssymb}
\usepackage[dvipdfmx]{graphicx}
\usepackage{enumitem}
\usepackage[nobreak]{cite}
\usepackage{float}


\begin{document}

\title{\LaTeX 卒論}
\author{横田祐汰}
\date{\today}
\maketitle

% \begin{abstract}

% \end{abstract}

% \tableofcontents

\chapter{はじめに}
\section{研究背景}
昨今、環境への配慮や健康増進の観点から移動手段としての
自転車利用が推進されている。
政府は平成29年度に自転車活用推進法を施行して以来、
自転車利用の促進を図るとともに自転車・歩行者・自動車が安全に通行できるよう
道路環境整備に関する指針を示してきた。
しかし、これまでに実施された道路環境整備の多くは既存の車道の路肩に線を引いて
自転車の通行位置を示すにとどまる車道混在型の整備であった。

その背景として政府は自転車道の設置に必要とされる幅員(2m以上)を確保できない道路が多く、
専用自転車道の整備が困難な状況が多数生じている点を挙げている。
そこで政府は令和元年に道路構造令を改正し、
自転車専用通行帯という新たな道路の形態を規定することで
限られた道路空間においても歩行者の安全を確保しながら
道路環境の整備を行うことのできる選択肢を提示した。
自転車専用通行帯とは自転車のみが通行を許可されている車両通行帯の事である。
しかし、令和6年に政府が公表したデータ[1]
%https://www.mlit.go.jp/road/road/bicycle/pdf/point.pdf
によると整備された道路約7500kmのうち、9割以上にも
のぼる約6600kmが車道混在型の整備手法を採用している。
政府は新たな道路形態を提示し,複数の整備手法の中から
選択可能な状況を整えているにもかかわらず
実際の整備は車道混在型に偏っているという状況が生じている。
この要因として道路拡幅が容易に行えないという物理的な制約に加え、
専用自転車道、自転車専用通行帯、歩行者自転車道といった複数の道路の形態の中から、
どの方式を選択するべきであるか明確でない点が挙げられる。
それは、それぞれの整備手法を選択した場合に生じる利点や
課題について検討した研究が少ないことに起因していると考えられる。
% また、それぞれの整備手法を選択した場合に生じる利点や課題について検討した研究が少ないため、
% 費用や時間を要しない車道混在型の整備が暫定的な措置として選択されている可能性も考えられる。
\section{研究目的}
本論文の目的は地方部に多く存在する第三種道路の片側一車線道路を対象として、
道路拡幅を伴わない再配分による自転車が安全に走行するための整備方法を
交通シミュレーションを使用して多角的に比較・評価することで、
道路の整備形態が交通流に与える影響を明らかにすることである。
再構築を行うことで生じる自転車・自動車・歩行者の安全性や快適性への影響を定量的に分析し、
安全性や十分な走行速度の確保など、整備目的に応じた適切な道路設計手法の選択に資する知見を得ることを目指す。

\chapter{関連研究}

\section{交通シミュレーションに関する研究}
これまで自転車エージェントを使用して交通流を
シミュレーションした研究は数多く行われている。
文献[1]
%https://www.jstage.jst.go.jp/article/pjsai/JSAI2023/0/JSAI2023_1F3GS505/_pdf/-char/ja 
秋山らモデル歩行者と電動スクーター、セグウェイ
のエージェントを構築して混合交通シミュレーションを行った[1]。
この論文では低密度と高密度それぞれについて歩行者流を
作り、低密度の歩行者流ではパーソナルモビリティが歩行者を追い越しながら移動する
様子が観測された一方、高密度の歩行者流では歩行者を追い越すことが困難になるなど
現実に即した挙動が観測されている。

実際に発生している交通問題の解決を目的に交通シミュレーションを活用した研究として次の論文が挙げられる。
文献[2]
% https://www.skr.mlit.go.jp/kikaku/kenkyu/r2/ronbun/I-41.pdf
大西らは、香川県丸亀市の実在する道路を対象として
交通シミュレーションにより交通状況を再現し、
渋滞発生の要因となり得る因子の調査を行っている[2]。
文献[3]
% https://www.jstage.jst.go.jp/article/jscejj/79/7/79_22-00200/_pdf/-char/ja
大井らは交通状況ついて実データを集計し、それをもとに
ミクロ交通シミュレーションを作成することで、赤信号・青信号それぞれの
オフセット時間の最適なパターンについて調査を行っている[3]。

これらの研究のように、マルチエージェントシミュレーション(以下、MASとする)を用いて
交通シミュレーションを行った研究は多く行われており、現実世界で起こっている
交通問題の解決に広く役立てられている。

\section{自転車走行空間整備に関する研究}
自転車が路肩を走行する危険性について扱った研究も多く行われている。
% https://www.jstage.jst.go.jp/article/jsaeronbun/44/1/44_20134034/_pdf/-char/en
紫藤は自転車が車道を走る車道混在型の危険性について心理的な観点から調査を行った。
まず、自転車利用者に対してアンケートを行い
車道の走行についてのどのように感じているか調査を行った[4]。
次に、ドライブシミュレーターを用いて路肩を走行している自転車の追い越しを、
車道幅員を変化させながら被験者に行ってもらい、
どの程度危険に感じたか評価してもらった。
2つの実験の結果、自転車は車道の走行に危険意識を感じており、
自動車の運転手は速度が速いほど、自転車との距離が短いほど
危険に感じることが分かった。
% https://www.jstage.jst.go.jp/article/jscejipm/72/5/72_I_1095/_pdf/-char/ja
横関は千葉県東葛地域の交通事故統計データを用いて歩道走行と比較した車道走行(左側通行)の危険性
について検証している[5]。
この検証の結果、車道走行の方が危険性が高いことが判明し、
自転車専用帯の設置が提言された。

これらの研究の結果から、自転車が車道の路肩を走行するのは
心理的にも、客観的な事故件数からも危険であり自転車道の整備
の必要性がわかる。

\section{道路の再構築・再配分に関する研究}
齊藤らは、ミクロな道路空間の再構築を検討した研究を行い、道路空間の再構築の是非を検証した[6]。
% https://www.jstage.jst.go.jp/article/jscejipm/78/5/78_I_671/_pdf/-char/ja
この研究では、両側6車線を有する国道1号線の一部区間を対象とし、
自転車通行空間の整備に関する検討を行っている。
対象区間は、自転車および歩行者の交通量が多いにもかかわらず、
安全に共存できる走行環境が十分に整備されていないという課題を抱えていた。
そこで、道路空間の再配分による自転車通行環境の改善を目的として、
交通シミュレーションを用いた比較検討が行われた。
単路部では、車線を1車線削減し、自転車専用通行帯を設ける再構築案を設定し、
現況との比較を行っている。
また、交差点部においては、現況の車道混在型と車線再配分を伴い自転車専用通行帯を導入した案の2種類を対象として、
交通シミュレーションにより評価を行っている。
評価指標としては、所要時間、旅行速度に加え、事故になりえる潜在的なリスクの発生頻度を計測するSSAMをを用いて、
再構築による交通影響および道路空間の安全性の変化を比較している。

文献[7]
% http://library.jsce.or.jp/jsce/open/00039/201105_no43/pdf/388.pdf
中野らは自動車の交通量が多いため、自転車の約98%が歩道を走行している
道路を対象にして、現状を改善するために自転車道の4つの案を区間ごとに道路に施して
交通実態調査、アンケート調査、街頭インタビューを行うことで
総合的な評価が高い案を調査している[7]。

既存の幅員から拡幅などを行わず、道路空間の再配分のみを前提として、
適切な道路再構築手法について検討した研究も存在する。
文献[8]
% https://www.jstage.jst.go.jp/article/jsteproceeding/42/0/42_729/_pdf/-char/ja
三好らは、道路全体の幅員を固定させ、道路構造令に従ったカーブサイドの整備方法について
停車量や交通量を変化させながら交通シミュレーションを用いて
比較検討を行い、交通条件ごとのカーブサイドの最適な整備方法の提言を行った[8]。

これらの研究から道路の再構築を行う方法として実データを用いて
MASを用いたミクロな交通シミュレーションを
行うという方法や、道路の一部を一時的に再編することで、どの再構築方法が
客観的な指標や、住民の心理的な評価が高いかを調査して道路の
編成方法を決定するという手法があることが分かった。




\section{本論文の位置づけと新規性}
これまでの研究から自転車が車道を走行する危険性や、
MASを用いた交通シミュレーションが道路の再構築
方法を検討するうえで有効であることなどが分かった。
一方で自転車エージェントを用いてシミュレーションを行い、
道路の再構築のベストプラクティスを検討した研究は少ない。
その上、ほとんどが都市部を対象とした片側複数車線の道路を
扱った研究である。
そこで、本論文ではこれまで扱われてこなかった
再拡幅が行えない第三種・片側一車線道路を
対象とすることで、地方の道路で再整備を行う際の一つの判断材料と
なることを目指す。


\chapter{エージェントの概要}
\section{シミュレーション全体構成}
本論文では、GAMAをプラットフォームとして使用して実験環境を構築する。
GAMAは定義や挙動の記述を行うことで
エージェントを作成することができる。
そのエージェントをmainファイルで呼び出して
インスタンス作成することで、エージェント同士が有機的に作用しあうMASを
作成することができる。
本論文の実験では交差点エージェント、車道エージェント、歩道エージェント、自動車エージェント、自転車エージェント、歩行者エージェントの6つを
交通シミュレーションに使用する。自転車エージェントはソーシャルフォースモデルをもとに作成し、
道路エージェント、歩行者エージェント、自動車エージェントはGAMA上の既存のエージェントを改良し実験に利用する。
本実験では以下のステップで行う。
\begin{enumerate}
  \item 交差点エージェント、車道エージェント、歩道エージェントを初期化する
  \item 自動車エージェント、自転車エージェント、歩行者エージェントを生成し、走行させる
  \item 計測をする
\end{enumerate}

%本実験の環境の説明
%図の挿入
\newpage
%説明の続き


\section{ソーシャルフォースモデルの概要}
ソーシャルフォースモデルはDirkHelbingとP'eter Moln'arが文献[9]
% https://arxiv.org/pdf/cond-mat/9805244
で提唱したモデルで、交通シミュレーションなど回避を伴うシミュレーションを
行う際に利用されることが多い。これはエージェントを目標地点へ向かう引力、
他のエージェントから受ける斥力、道路の境界から受ける斥力に分解して個別に計算した
結果を足し合わせることでエージェントの行動を再現するモデルである。
このモデルを改良して歩行者エージェントを作成し、その後歩行実験を行ってモデルの
精度をQ\_K図にて検証した実験では1平方メートル内に4人の歩行者が存在する
密度以下の環境では高い精度で歩行者を再現できていることが分かった[1]。
ソーシャルフォースモデルは複雑な交通流を再現する場合においても
各エージェントに対して行動規則を定義することで
エージェントが自律的に挙動するモデルであり
マルチエージェントシミュレーションの構築に非常に適している。
そこで本論文では、エージェントの構築に
ソーシャルフォースモデルを採用した。


\section{エージェントの構築と概要}
\subsection{自転車エージェント}
自転車エージェントモデルをもとに、一部改良を加えて作成した。ソーシャルフォースモデルの欠点として、
双方向からのすれ違いが発生するシミュレーションにおいて
衝突が起きやすいことが挙げられる。
双方向のすれ違いが発生するシミュレーションにおいて衝突が生じやすい点が挙げられる。
これは斥力の作用メカニズムに起因する。
一般に他のエージェントから受ける斥力は
他のエージェントの現在位置と $\Delta t$ 秒後の予測位置の双方を考慮し、
それぞれの位置から自分の位置へのベクトルを算出したうえで、
その平均ベクトルの向きを斥力方向として定める。
しかし自身の進行方向上に対向して走行するエージェントが存在する場合、
斥力が進行方向とほぼ180度反対の方向に作用し、回避ではなく減速方向にのみ作用してしまう。
その結果十分な側方回避が行われず、衝突が発生しやすくなる。
そこで本研究では進行方向と斥力が打ち消し合う向きに働く場合に
進行方向と垂直なベクトルを追加し、側方回避を促すように拡張した。
追加する方向は相手エージェントが
自分より左側に位置する場合は右方向に、右側に位置する場合は左方向とする。
また、双方が同じ位置にある場合は左右のいずれかをランダムに選択する。
本エージェントは出発点や目的地に加え、ソーシャルフォースモデルで用いる
パラメータを引数として与えることで、加速度や自転車・歩行者・自動車・道路境界から受ける
斥力の強さを決定する。

\subsection{自動車エージェント}
自動車エージェントはGAMAが提供しているdrivingスキルをもとに作成を行った。
本エージェントは速度や加速度という走行速度に関する情報と、
出発点や目的地や使用する道路という走行場所に関する情報
を引数に設定することで、走行を実現している。
本実験では自動車エージェントは他の自動車エージェントであれば斥力が働くが、
自転車や歩行者に対しては斥力が働かないものとする。

\subsection{歩行者エージェント}
歩行者エージェントはGAMAが提供しているpedestrianスキルをもとに作成を行った。
本エージェントは速度と出発点、目的地
を引数に設定することで、走行を実現している。
また、回避対象を設定することができる。

\subsection{車道エージェント}
車道エージェントは車両が通過する箇所の道路設定や描画処理を行う。
なお、自転車が自転車専用道路や自転車歩行者道を走行する際は例外として、歩道エージェントが描画処理を行う。
本エージェントは道路の中心点、センターラインの幅員、車道の幅員、路肩の幅員、自転車専用通行帯
を引数として受け取ることで車道の構成を決定している。

\subsection{歩道エージェント}
歩道エージェントは歩行者が通過する箇所の道路設定や描画処理を行う。
先述した通り自転車が自転車専用道路や自転車歩行者道を走行する際は例外として、
歩道エージェントで描画した箇所を走行する。
本エージェントは道路の中心点、ガードレールの幅員、自転車専用道路の幅員、植樹帯の幅員、歩行者専用道の幅員、歩行者自転車道の幅員
を引数として受け取ることで歩道部の構成を決定している。
また、引数の情報をもとに歩行者が自由に動くことのできる空間を設定することで
歩行者が2次元平面上を移動できるようになり、すれ違いなどの挙動を可能にしている。

\subsection{交差点エージェント}
交差点エージェントはGAMAが提供しているintersectionスキルをもとに作成を行った。
本エージェントは道路の端点を決定するエージェントである。
今回の実験では交差点部ではなく単路部を対象にした実験であるため、
信号処理の実装は行わない。


\section{自転車エージェントのパラメータ設定}
ソーシャルフォースモデルをもとに、自転車エージェントの作成を行うにあたって、
使用するパラメータを選定する必要がある。ソーシャルフォースエージェントの作成に
必要なパラメータは下図の4つである。
そのうち、$\tau$と$\lambda$は
自転車エージェント単独で決定する値である。
A,Bの値は自転車と他のエージェントとの相互作用によって
用いる値であり、相手のエージェントの種類に依存する値である。
今回の実験では、対自転車、対歩行者、対自動車、対道路の境界の
4種類のパラメータを用意する。
本来は、交通シミュレーションを行う対象の道路にて動画を撮影し、
自転車の軌跡を記録する。
その軌跡と誤差がなるべく生じないようパラメータを
設定するという手法が多く用いられているが、今回の実験では
時間や機材が不足しているためその手順を踏むことができなかった。
そこで、今回の実験では他の似たような実験環境で使用されている
論文のパラメータをそのまま採用する。
本論文では、Quらの研究の値を使用することにした[9]。
この論文では、中国鄭州市の単路部における
電動自転車–自動車混合交通を対象にビデオ撮影による実軌跡データを収集し、
モデルの軌跡との誤差が最小限になるように最尤推定法を用いて
ソーシャルフォースモデルのパラメータを推定している。
ただし、この研究では自転車が歩行者を回避する際に使用するパラメータを
もとめていない。
そこで、自転車が歩行者を回避する際に用いるパラメータについては
Helbing が提唱したソーシャルフォースモデルの原論文を参照する[10]。
この研究では,歩行者同士が相互に回避行動をとる際に用いられる
パラメータが示されている。
本研究では、平地単路における歩行者回避という条件が近いため、
これらの値を初期パラメータとして採用した。


% \begin{table}[H]
% \centering
% \begin{tabular}{|c|}
% \hline
% $\tau$ \\ \hline
% $\lambda$ \\ \hline
% A \\ \hline
% B \\ \hline
% \end{tabular}
% \caption{SFMで使用するパラメータ}
% \label{tab:one_col_six_rows}
% \end{table}

\begin{table}[H]
\centering
\begin{tabular}{|c|c|}
\hline
$\tau$&9 \\ \hline
$\lambda$&0.3 \\ \hline
A\_pedestrian&2 \\ \hline
B\_pedestrian&0.5 \\ \hline
\end{tabular}
\caption{対歩行者に使用するパラメータ}
\label{tab:one_col_six_rows}
\end{table}

% 自転車が自転車を回避するために設定するパラメータは
% Liらの論文
% %https://www.sciencedirect.com/science/article/abs/pii/S1569190X20301921
% を参考に使用する[9]。
% この研究では
% 本研究で参考とした先行研究は、対象が交差点部であり、
% 本論文で扱う単路部とは構造に違いがあるものの、
% 平面空間における自転車同士の相互作用をミクロな視点で
% 扱っている点で共通している。
% また、ラッシュアワー時を対象として軌跡データを収集・分析
% しており、交通密度が高い状況下での自転車挙動が反映されている。
% さらに、分析に用いられたサンプル数も743台と十分に多く、
% 統計的な信頼性も高い。以上の理由から、当該研究で示されている
% パラメータおよびモデル構造は、本論文におけるシミュレーション
% 研究においても十分に適用可能であると判断した。

\begin{table}[H]
\centering
\begin{tabular}{|c|c|}
\hline
$\tau$&9 \\ \hline
$\lambda$&0.3 \\ \hline
A\_bike&0.42 \\ \hline
B\_bike&8.04 \\ \hline
\end{tabular}
\caption{対自転車に使用するパラメータ}
\label{tab:one_col_six_rows}
\end{table}





% 自転車が車両を回避するために使用するパラメータは
% Zhao-weiらの論文の値
% %https://journals.sagepub.com/doi/epub/10.1177/1687814017719641
% を使用する[11]。
% この論文内では、自転車と車両が相互作用する環境を扱っている。

\begin{table}[H]
\centering
\begin{tabular}{|c|c|}
\hline
$\tau$&9 \\ \hline
$\lambda$&0.3 \\ \hline
A\_car&8 \\ \hline
B\_car&12 \\ \hline
\end{tabular}
\caption{対自動車に使用するパラメータ}
\label{tab:one_col_six_rows}
\end{table}

% 自転車が道路の境界から受ける斥力を求めるために使用するパラメータは

\begin{table}[H]
\centering
\begin{tabular}{|c|c|}
\hline
$\tau$&9 \\ \hline
$\lambda$&0.3 \\ \hline
A\_road&2.3\\ \hline
B\_road&5\\ \hline
\end{tabular}
\caption{道路の斥力に使用するパラメータ}
\label{tab:one_col_six_rows}
\end{table}

\chapter{実験}
\section{実験目的}
本論文の実験では,道路の拡幅や車線削減によって自転車道を新設することが困難な道路を対象とし,
そのような道路において実施可能な再構築手法が
自転車,歩行者,および自動車の安全性および快適性に
どのような影響を及ぼすのかについて交通シミュレーションを用いて
定量的に評価することを目的とする。

\section{実験環境}
本論文では幅員16mの第三種三級道路を対象とする。
第四種道路を使用せず
第三種道路を選定した理由として道路の車線数は交通量に応じて決定されるため、
都市部の道路や高速自動車道自動車専用道路では
交通量が多く片側一車線道路の割合が低くなる点が挙げられる。
一方、地方部に多く存在する第三種道路では交通量が比較的少ないため、
片側一車線道路が広く分布しており本論文の対象として適していると考えられる。
また、第三種道路の中でも交通量が多い道路ほど事故発生のリスクが高く、
再構築による安全性向上の必要性が大きいと考えられることから
本論文では第三種第二級道路を対象とした。
第三種第一級を採用しない理由としては、道路構造令第5条により車線数が4以上であることが
基本とされているためである。
幅員を16 mとした理由は、安全で快適な自転車利用環境創出ガイドラインにおいて、
片側一車線道路の代表的な例として16m幅員の道路が示されているためであり、
本論文ではこの条件を実験環境として採用する。
なお、交差点部は信号制御や交差交通流などネットワーク構造の影響を強く受けるため、
本論文ではそれらの影響を排除し道路構造および再配分方法そのものの
効果を明確に評価する目的から単路部を実験対象とした。
先述した通り、再構築前の道路として安全で快適な自転車利用環境創出ガイドラインに
示されていた道路をそのまま利用する。

\section{実験設定}
本論文では以下に示す再構築前の道路と5つの道路設計で
再構築後の道路を交通シミュレーションにより比較する。
なお、すべての再構築後の道路設計で道路構造令を遵守している。
\subsection{実験設定1 再構築前の道路}
再構築前の道路として幅員が16mで長さ100mの直線道路を使用する。
本論文では安全で快適な自転車利用環境創出ガイドラインの例を採用する。
計画交通量は1日当たり4754台とし、これは国土交通省が発表している
令和3年度全国道路・街路交通情勢調査  一般交通量調査
の結果を利用した。
規制速度は60km/hとし、ガイドラインで示されている設計速度の数値を使用する。
なお、停車は行うことができない区域を対象とする。

\includegraphics[width=15cm]{../設定1.png}
%自転車はすべて所定の道路を進むようにする。など細かい注意書き

\subsection{歩行者の走行}
政府が発表している資料では、1日1500台以上走行している道路に対して
交通量が多いと言えると目安が示されている。
よって本実験でもこれを採用する。
なお、歩行者は歩道を双方向に通行することができる。
歩行者は歩道の端点からランダムな位置で生成する。
今回の実験設定では自転車や自動車とは植樹帯によって物理的に分断されているため、
他の歩行者と道路の境界以外からは斥力を受けないものとする。

\subsection{自転車の走行}
政府が発表している資料では、1日500台以上走行している道路に対して
交通量が多いと言えると目安が示されている。
よって本実験でもこれを採用する。
57/sに一度スポーンさせる。
自転車は始点の路肩にランダムにスポーンさせる。
また、自動車と同様に一方通行しか行うことができない。
今回の実験では自転車と自動車から斥力を受けるように設定する。
また、道路の斥力は車道全体の境界から受けるものとし、路肩の境界からは受けないものとする。


\subsection{実験設定2 車道に速度制限をかける}
実験設定2では車道の速度制限を60km/hから40km/hに変更する。
ガイドラインにて示されている交通状況を踏まえた整備形態の選定(完成形態)の考え方 
にて、規制速度の強化や交通量の減少が計画できる場合は車道混在のままで完成形態
とすることができる。
この40km/hという速度は第三種第二級道路の設計速度のうち、
選択しうる中で最低の速度である。
交通量の減少を行うためには迂回路を作成するなどの手段を構築することが考えられるが
ミクロシミュレーションの域を超えているため本論文では行わない。
そのほか、道路の構成は再構築前の道路と統一する。
自転車や歩行者の走行も


\subsection{歩行者の走行}
政府が発表している資料では、1日1500台以上走行している道路に対して
交通量が多いと言えると目安が示されている。
よって本実験でもこれを採用する。
57/sに一度スポーンさせる。
なお、歩行者は歩道を双方向に通行することができる。
歩行者は歩道の端点からランダムな位置で生成する。
今回の実験設定では自転車や自動車とは植樹帯によって物理的に分断されているため、
歩行者や道路の境界以外からは斥力を受けないものとする。

\subsection{自転車の走行}
政府が発表している資料では、1日500台以上走行している道路に対して
交通量が多いと言えると目安が示されている。
よって本実験でもこれを採用する。
57/sに一度スポーンさせる。
自転車は始点の路肩にランダムにスポーンさせる。
また、自動車と同様に一方通行しか行うことができない。
今回の実験では自転車と自動車から斥力を受けるように設定する。
また、道路の斥力は車道全体の境界から受けるものとし、路肩の境界からは受けないものとする。



\subsection{実験設定3 自転車専用通行帯の設置}
実験設定3では自転車専用通行帯を設置し自転車が車道の路肩を走らなくていいように区分を行う。
そのためセンターラインの省略を行い自転車専用通行帯の幅員を確保する。
また、道路構造令第8条第7項より路肩を設けず、その幅員を縮小する。
そうすることで車道内に幅員1.5mの自転車専用通行帯を設置する。
本シミュレーションでは、歩行者は車道と植樹帯によって物理的に隔離されているため
自転車や自動車からの影響は受けない。
また、自転車と自動車は自転車専用通行帯の境界線から斥力を受ける。

\includegraphics[width=15cm]{./setting3.png}

\subsection{歩行者の走行}
同様に57/sに一度スポーンさせる。
なお、歩行者は歩道を双方向に通行することができる。
歩行者は歩道の端点からランダムな位置で生成する。
今回の実験設定では自転車や自動車とは植樹帯によって物理的に分断されているため、
歩行者や道路の境界以外からは斥力を受けないものとする。

\subsection{自転車の走行}
政府が発表している資料では、1日500台以上走行している道路に対して
交通量が多いと言えると目安が示されている。
よって本実験でもこれを採用する。
57/sに一度スポーンさせる。
自転車は始点の自転車専用通行帯にランダムにスポーンさせ、実験設定1
と同様に一方通行しか行うことができない。
今回の実験では道路の分断はされているが、物理的に分断されていないため
自動車と自転車から斥力を受けるものとする。
また、道路の斥力は自転車専用通行帯から受ける。


\subsection{実験設定4 自転車専用道路の設置}
実験設定4では自転車専用道路の設置を行い、自転車と車道を物理的に区分する。
実験設定3と同様にセンターラインの省略、路肩の省略を行い自転車専用道路の幅員を確保する。
なお、やむを得ない場合は自転車専用道路の幅員は1.5mに削減することができるとあり、
これを適用する。
また、植樹帯を1.5mから1mに削減し、余った空間に車道と自転車専用道路を隔てる0.5mの物理的な
壁を設ける。
また、歩道と自転車専用道路を隔てるように植樹帯を設置する。これを行う根拠として、
自転車は車両に分類されるため、区分でいうと歩行者と自転車、自動車という位置づけになるため。
この整備により車道と自転車専用道路と歩道は物理的に遮断されるためそれぞれが他の乗物による
斥力が発生しないものとする。

\includegraphics[width=20cm]{./設定4.png}

\subsection{歩行者の走行}
同様に57/sに一度スポーンさせる。
なお、歩行者は歩道を双方向に通行することができる。
歩行者は歩道の端点からランダムな位置で生成する。
今回の実験設定では自転車や自動車とは植樹帯によって物理的に分断されているため、
歩行者や道路の境界以外からは斥力を受けないものとする。

\subsection{自転車の走行}
同様に57/sに一度スポーンさせる。
自転車は始点の自転車専用道路にランダムにスポーンさせ、実験設定1
と同様に一方通行しか行うことができない。
今回の実験では歩道と車道ともに物理的に分断されているため、自転車のみに
斥力を受ける。
また、道路の斥力は自転車専用通行帯から受ける。


\subsection{実験設定5 自転車歩行者道に変更}
実験設定5では歩道を自転車歩行者道に変更し、拡張する。
センターラインの省略、路肩の省略、植樹帯の1.5mから1mへの削減を行い自転車歩行者道の幅員を確保する。
自転車歩行者道での対応という資料にのっとり、
自転車歩行者道の幅員を4m確保し、歩道上で自転車の双方向通行を許容する。

\includegraphics[width=20cm]{./設定5.png}

\subsection{歩行者の走行}
同様に57/sに一度スポーンさせる。
なお、歩行者は歩道を双方向に通行することができる。
歩行者は歩道の端点からランダムな位置で生成する。
今回の実験設定では自転車と歩行者と道路の境界からは斥力を受けるものとする。

\subsection{自転車の走行}
同様に57/sに一度スポーンさせる。
自転車は左右の両端から歩道にランダムにスポーンさせ、
双方向に通行することができる。
今回の実験では車道と物理的に分断されているため、歩道と自転車から
斥力を受ける。
また、道路の斥力は歩道の境界から受ける。


% \subsection{自転車の走行}
% 政府が発表している資料では、1日500台以上走行している道路に対して
% 交通量が多いと言えると目安が示されている。
% よって本実験でもこれを採用する。
% 172/sに一度スポーンさせる。

% \subsection{歩行者の走行}
% 政府が発表している資料では、1日8000人以上走行している道路に対して
% 交通量が多いと言えると目安が示されている。
% よって本実験でもこれを採用する。
% 20/sに一度スポーンさせる。



\subsection{自動車の走行}
本実験では使用する地方部の平地の国道における
第三種第二級道路の定義が1日当たりの交通量が20000台以下の道路である
ということから上限の20000台を1日に走行させる。
すなわち、4.32秒当たりに一台車をスポーンさせることで本実験ではその環境を作成する。

\section{評価指標}
本実験では,道路再構築が自転車および歩行者に与える影響を
安全性および快適性の観点から定量的に評価するため,
それぞれに対応した評価指標を設定する。

安全性の評価指標として,最近接距離を用いる。
最近接距離とはあるエージェントが道路区間を通過する過程において
他のエージェントと最も接近した際の距離を指す。
この指標を用いることで実際に事故として顕在化しない場合であっても、
潜在的に接触リスクが最も高まった瞬間を捉えることが可能となる。
そのため、事故件数のような長期間の観測を必要とせず、
交通シミュレーションにおいて安全性を評価する指標として有効であると考えられる。
本実験では自転車は自転車と自動車の最近接距離を測り、歩行者は自転車との最近接距離を
測る。
快適性の評価指標としては平均速度と通過時間を用いる。
平均速度は走行の円滑性や移動の効率性を表す基本的な指標であり、
自転車および歩行者がどの程度ストレスなく走行できているかを
間接的に評価することができる。
通過時間は生成されてから50/3mの道路を通過するまでの時間の平均をとる。
道路再構築によって通行空間が変化した際の
走行環境の改善あるいは悪化を定量的に把握するために、
平均速度と通過時間は適切な指標であると判断した。
各々の値の具体的な測定方法として、最近接距離はそれぞれのエージェントが生成されてから毎ステップごとに
他のエージェントとの距離を測定し、その値が最近接距離より小さかった場合に最近接距離の値を
更新する。そのようにして、エージェントが道路を通行し終わった際に、最近接距離を取り出し、
集計することで最近接距離の平均を求める。
平均速度は毎ステップごとに、ベクトルの絶対値をとり、採取したベクトルの数の合計で
割ることで値を求める。
通過時間は目的地に着いた時間から生成された時間を引き、採取したサンプルの合計で
割ることで値を求める。
また、十分に分散したデータをとるため、1000s後のデータを使用する。

% \vspace{10\baselineskip}

\section{実験結果}
本実験を行った結果、自転車と自動車の最近接距離は下図のような結果となった。
実験1と2の結果から走行速度を変更してもほとんど有意な差が生まれないことが分かった。
また、3の結果から自転車専用通行帯を設けることで自転車と自動車間の距離を確保がしやすくなり、
安全性を向上することができることが分かった。

% %最近接距離
\includegraphics[width=15cm]{自転車と自動車の距離.png}

% また、実験5で歩行者と自転車の最近接距離を求めた結果、6.67mという結果になった。

%平均速度
自転車の平均速度は下図のような結果になった。実験1と2の結果から速度の制限を行っても
ほとんど有意な差が生じないことがわかる。また、実験1と2と3では自転車の速度にほとんど変化が見られない。
実験1,2,3と実験4,5では速度に大きな差がみられる。
また、実験4と5では有意な差が見られなかった。
\includegraphics[width=15cm]{自転車速度.png}

% %通過時間
自転車の通過時間は下図のような結果になった。
実験1、2、3では、実験3だけわずかに通過時間が短縮されている。また、実験1,2,3と実験4,5では
通過時間に大きな差がみられる。その中でも実験4と5では実験4の方が通過時間が短いことが分かった。

\includegraphics[width=15cm]{通過時間.png}

\chapter{考察}
今回の実験から、自転車の安全性や快適性は自転車と車両が物理的に分断されて
いるような道路の形状だと、向上することが分かった。
また、自転車が車道を走行する場合でも自転車専用通行帯を作成することで
自転車の速度はほとんど変わらないが、自転車の通過時間がやや短くなることが分かった。
このような事象が起きた要因として、自転車が車両を回避する動きが減少することで
進行方向の動きが大きくなったことが考えられる。このことから、安全性も快適性も
自転車専用通行帯を作成した方が向上する。
自転車が車道以外を走る場合は、自転車が自転車専用道を走行した場合は
自転車の速度は変わらないが、通過時間がわずかに短くなった。この結果から、
自転車が歩行者を回避する動きが減少した結果だと思われる。
よって、自転車のために物理的に分断されている道路を走行する場合は、
自転車歩行者道を設けるのではなく、自転車専用道を設けた方がよいことが分かった。
しかし、今回のシミュレーションではすべての自転車が既定の箇所を走行することを
前提とした実験である。そのため、自転車専用道などの幅員が狭い道路で逆走が
起こった場合は快適性や安全性が著しく低下することが予想される。
その一方、今回のシミュレーションで歩行者道は双方向通行可能としたため、
安全性や快適性に大きな違いが出るとは考えずらい。
そこで、幅員が16mの第三種第二級道路で再配分による再構築を行う場合、
自転車専用道の走行を行うことが




\chapter{おわりに}
% \vspace{10\baselineskip}


\begin{thebibliography}{10}
\item 塚原壮平, 「初めてのLaTeX講座」, 九大新書, 2023.
\item 塚原壮平, 「素粒子物理学入門」, きゅうり図書, 2023.
\item https://www.mlit.go.jp/road/sign/kijyun/pdf/20190705tuukoutai.pdf
\end{thebibliography}

\end{document}