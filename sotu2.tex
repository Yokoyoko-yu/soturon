\documentclass[a4paper, 11pt]{jsreport}
\usepackage{amsmath}
\usepackage{amssymb}
\usepackage[dvipdfmx]{graphicx}
\usepackage{enumitem}
\usepackage[nobreak]{cite}
\usepackage{float}


\begin{document}

\title{\LaTeX 卒論1}
\author{横田祐汰}
\date{\today}
\maketitle

% \begin{abstract}

% \end{abstract}

% \tableofcontents

\chapter{はじめに}
\section{研究背景}
昨今、環境への配慮や健康増進の観点から移動手段としての
自転車利用が推進されている。
政府は平成29年度に自転車活用推進法を施行して以来、
自転車利用の促進を図るとともに自転車・歩行者・自動車が安全に通行できるよう
道路環境整備に関する指針を示してきた。
しかし、これまでに実施された道路環境整備の多くは既存の車道の路肩に線を引いて
自転車の通行位置を示すにとどまる車道混在型の整備であった。

その背景として政府は自転車道の設置に必要とされる幅員(2m以上)を確保できない道路が多く、
専用自転車道の整備が困難な状況が多数生じている点を挙げている。
そこで政府は令和元年に道路構造令を改正し、
自転車専用通行帯という新たな道路の形態を規定することで
限られた道路空間においても歩行者の安全を確保しながら
道路環境の整備を行うことのできる選択肢を提示した。
自転車専用通行帯とは自転車のみが通行を許可されている車両通行帯の事である。
しかし、令和6年に政府が公表したデータ[1]
%https://www.mlit.go.jp/road/road/bicycle/pdf/point.pdf
によると整備された道路約7500kmのうち、9割以上にも
のぼる約6600kmが車道混在型の整備手法を採用している。
政府は新たな道路形態を提示し,複数の整備手法の中から
選択可能な状況を整えているにもかかわらず
実際の整備は車道混在型に偏っているという状況が生じている。
この要因として道路拡幅が容易に行えないという物理的な制約に加え、
専用自転車道、自転車専用通行帯、歩行者自転車道といった複数の道路の形態の中から、
どの方式を選択するべきであるか明確でない点が挙げられる。
その原因として、整備手法を選択した場合に生じる利点や
課題について検討した研究が少ないことが考えられる。

\section{研究目的}
本論文の目的は地方部に多く存在する第三種道路の片側一車線道路を対象として、
道路拡幅を伴わない再配分による自転車道の整備方法を
交通シミュレーションを使用して比較・評価することで、
道路の整備形態が交通流に与える影響を明らかにすることである。
再構築を行うことで生じる自転車の安全性と快適性への影響を定量的に分析し、
安全性や十分な走行速度の確保など、整備目的に応じた適切な道路設計手法の選択に
資する知見を得ることを目指す。

\chapter{関連研究}

\section{交通シミュレーションに関する研究}
これまで自転車エージェントを使用して交通流を
シミュレーションした研究は数多く行われている。
文献[1]
%https://www.jstage.jst.go.jp/article/pjsai/JSAI2023/0/JSAI2023_1F3GS505/_pdf/-char/ja 
秋山らモデル歩行者と電動スクーター、セグウェイ
のエージェントを構築して混合交通シミュレーションを行った[1]。
この論文では低密度と高密度それぞれについて歩行者流を
作り、低密度の歩行者流ではパーソナルモビリティが歩行者を追い越しながら移動する
様子が観測された一方、高密度の歩行者流では歩行者を追い越すことが困難になるなど
現実に即した挙動が観測されている。

実際に発生している交通問題の解決を目的に交通シミュレーションを活用した研究として次の論文が挙げられる。
文献[2]
% https://www.skr.mlit.go.jp/kikaku/kenkyu/r2/ronbun/I-41.pdf
大西らは、香川県丸亀市の実在する道路を対象として
交通シミュレーションにより交通状況を再現し、
渋滞発生の要因となり得る因子の調査を行っている[2]。
文献[3]
% https://www.jstage.jst.go.jp/article/jscejj/79/7/79_22-00200/_pdf/-char/ja
大井らは交通状況ついて実データを集計し、それをもとに
ミクロ交通シミュレーションを作成することで、赤信号・青信号それぞれの
オフセット時間の最適なパターンについて調査を行っている[3]。

これらの研究のように、マルチエージェントシミュレーション(以下、MASとする)を用いて
交通シミュレーションを行った研究は多く行われており、現実世界で起こっている
交通問題の解決に広く役立てられている。

\section{自転車走行空間整備に関する研究}
自転車が路肩を走行する危険性について扱った研究も多く行われている。
% https://www.jstage.jst.go.jp/article/jsaeronbun/44/1/44_20134034/_pdf/-char/en
紫藤は自転車が車道を走る車道混在型の危険性について心理的な観点から調査を行った。
まず、自転車利用者に対してアンケートを行い
車道の走行についてのどのように感じているか調査を行った[4]。
次に、ドライブシミュレーターを用いて路肩を走行している自転車の追い越しを、
車道幅員を変化させながら被験者に行ってもらい、
どの程度危険に感じたか評価してもらった。
2つの実験の結果、自転車は車道の走行に危険意識を感じており、
自動車の運転手は速度が速いほど、自転車との距離が短いほど
危険に感じることが分かった。
% https://www.jstage.jst.go.jp/article/jscejipm/72/5/72_I_1095/_pdf/-char/ja
横関は千葉県東葛地域の交通事故統計データを用いて歩道走行と比較した車道走行(左側通行)の危険性
について検証している[5]。
この検証の結果、車道走行の方が危険性が高いことが判明し、
自転車専用帯の設置が提言された。

これらの研究の結果から、自転車が車道の路肩を走行するのは
心理的にも、客観的な事故件数からも危険であり自転車道の整備
の必要性がわかる。

\section{道路の再構築・再配分に関する研究}
齊藤らは、ミクロな道路空間の再構築を検討した研究を行い、道路空間の再構築の是非を検証した[6]。
% https://www.jstage.jst.go.jp/article/jscejipm/78/5/78_I_671/_pdf/-char/ja
この研究では、両側6車線を有する国道1号線の一部区間を対象とし、
自転車通行空間の整備に関する検討を行っている。
対象区間は、自転車および歩行者の交通量が多いにもかかわらず、
安全に共存できる走行環境が十分に整備されていないという課題を抱えていた。
そこで、道路空間の再配分による自転車通行環境の改善を目的として、
交通シミュレーションを用いた比較検討が行われた。
単路部では、車線を1車線削減し、自転車専用通行帯を設ける再構築案を設定し、
現況との比較を行っている。
また、交差点部においては、現況の車道混在型と車線再配分を伴い自転車専用通行帯を導入した案の2種類を対象として、
交通シミュレーションにより評価を行っている。
評価指標としては、所要時間、旅行速度に加え、事故になりえる潜在的なリスクの発生頻度を計測するSSAMをを用いて、
再構築による交通影響および道路空間の安全性の変化を比較している。

文献[7]
% http://library.jsce.or.jp/jsce/open/00039/201105_no43/pdf/388.pdf
中野らは自動車の交通量が多いため、自転車の約98%が歩道を走行している
道路を対象にして、現状を改善するために自転車道の4つの案を区間ごとに道路に施して
交通実態調査、アンケート調査、街頭インタビューを行うことで
総合的な評価が高い案を調査している[7]。

既存の幅員から拡幅などを行わず、道路空間の再配分のみを前提として、
適切な道路再構築手法について検討した研究も存在する。
文献[8]
% https://www.jstage.jst.go.jp/article/jsteproceeding/42/0/42_729/_pdf/-char/ja
三好らは、道路全体の幅員を固定させ、道路構造令に従ったカーブサイドの整備方法について
停車量や交通量を変化させながら交通シミュレーションを用いて
比較検討を行い、交通条件ごとのカーブサイドの最適な整備方法の提言を行った[8]。

これらの研究から道路の再構築を行う方法として実データを用いて
MASを用いたミクロな交通シミュレーションを
行うという方法や、道路の一部を一時的に再編することで、どの再構築方法が
客観的な指標や、住民の心理的な評価が高いかを調査して道路の
編成方法を決定するという手法があることが分かった。




\section{本論文の位置づけと新規性}
これまでの研究から自転車が車道を走行する危険性や、
MASを用いた交通シミュレーションが道路の再構築
方法を検討するうえで有効であることなどが分かった。
一方で自転車エージェントを用いてシミュレーションを行い、
道路の再構築のベストプラクティスを検討した研究は少ない。
その上、ほとんどが都市部を対象とした片側複数車線の道路を
扱った研究である。
そこで、本論文ではこれまで扱われてこなかった
再拡幅が行えない第三種・片側一車線道路を
対象とすることで、地方の道路で再整備を行う際の一つの判断材料と
なることを目指す。


\chapter{エージェントの概要}
\section{シミュレーション全体構成}
本論文では、GAMAをプラットフォームとして使用して実験環境を構築する。
GAMAは定義や挙動の記述を行うことで
エージェントを作成することができる。
そのエージェントをmainファイルで呼び出して
インスタンス作成することで、エージェント同士が有機的に作用しあうMASを
作成することができる。
本論文の実験では交差点エージェント、車道エージェント、歩道エージェント、自動車エージェント、自転車エージェント、歩行者エージェントの6つを
交通シミュレーションに使用する。自転車エージェントはソーシャルフォースモデルをもとに作成し、
道路エージェント、歩行者エージェント、自動車エージェントはGAMA上の既存のエージェントを改良し実験に利用する。
本実験では以下のステップで行う。
\begin{enumerate}
  \item 交差点エージェント、車道エージェント、歩道エージェントを初期化する
  \item 自動車エージェント、自転車エージェント、歩行者エージェントを生成し、走行させる
  \item 計測をする
\end{enumerate}


\section{ソーシャルフォースモデルの概要}
ソーシャルフォースモデルはDirkHelbingとP'eter Moln'arが文献[9]
% https://arxiv.org/pdf/cond-mat/9805244
で提唱したモデルで、交通シミュレーションなど回避を伴うシミュレーションを
行う際に利用されることが多い。これはエージェントを目標地点へ向かう引力、
他のエージェントから受ける斥力、道路の境界から受ける斥力に分解して個別に計算した
結果を足し合わせることでエージェントの行動を再現するモデルである。
このモデルを改良して歩行者エージェントを作成し、その後歩行実験を行ってモデルの
精度を検証した実験では1平方メートル内に4人の歩行者が存在する
密度以下の環境では高い精度で歩行者を再現できていることが分かった[1]。
ソーシャルフォースモデルは複雑な交通流を再現する場合においても
各エージェントに対して行動規則を定義することで
エージェントが自律的に挙動するモデルであり
マルチエージェントシミュレーションの構築に非常に適している。
そこで本論文では、エージェントの構築に
ソーシャルフォースモデルを採用した。


\section{エージェントの構築と概要}
\subsection{自転車エージェント}
自転車エージェントはソーシャルフォースモデルをもとに、一部改良を加えて作成した。
ソーシャルフォースモデルの欠点として、
双方向からのすれ違いが発生するシミュレーションにおいて
衝突が起きやすいことが挙げられる。
これは斥力の作用メカニズムに起因する。
一般に他のエージェントから受ける斥力は
他のエージェントの現在位置と $\Delta t$ 秒後の予測位置の双方を考慮し、
それぞれの位置から自分の位置へのベクトルを算出したうえで、
その平均ベクトルの向きを斥力方向として定める。
しかし自身の進行方向上に対向して走行するエージェントが存在する場合、
斥力が進行方向とほぼ180度反対の方向に作用し、回避ではなく減速方向にのみ作用してしまう。
その結果、十分な側方回避が行われず衝突が発生しやすくなる。
そこで本研究では進行方向と斥力が打ち消し合う向きに働く場合に
進行方向と垂直なベクトルを追加し、側方回避を促すように拡張した。
追加する方向は相手エージェントが
自分より左側に位置する場合は右方向に、右側に位置する場合は左方向とする。
また、双方が同じ位置にある場合は左右のいずれかをランダムに選択する。
本エージェントは出発点や目的地に加え、ソーシャルフォースモデルで用いる
パラメータを引数として与えることで、加速度や自転車・歩行者・自動車・道路境界から受ける
斥力の強さを決定する。

\subsection{自動車エージェント}
自動車エージェントはGAMAが提供しているdrivingスキルをもとに作成を行った。
本エージェントは速度や加速度という走行速度に関する情報と、
出発点や目的地や使用する道路という走行場所に関する情報
を引数に設定することで、走行を実現している。
本実験では自動車エージェントは他の自動車エージェントであれば斥力が働くが、
自転車や歩行者に対しては斥力が働かないものとする。

\subsection{歩行者エージェント}
歩行者エージェントはGAMAが提供しているpedestrianスキルをもとに作成を行った。
本エージェントは速度と出発点、目的地
を引数に設定することで、走行を実現している。
また、回避対象を設定することができる。

\subsection{車道エージェント}
車道エージェントは車両が通過する箇所の道路設定や描画処理を行う。
なお、自転車が自転車専用道路や自転車歩行者道を走行する際は例外として、歩道エージェントが描画処理を行う。
本エージェントは道路の中心点、センターラインの幅員、車道の幅員、路肩の幅員、自転車専用通行帯
を引数として受け取ることで車道の構成を決定している。

\subsection{歩道エージェント}
歩道エージェントは歩行者が通過する箇所の道路設定や描画処理を行う。
先述した通り自転車が自転車専用道路や自転車歩行者道を走行する際は例外として、
歩道エージェントで描画した箇所を走行する。
本エージェントは道路の中心点、ガードレールの幅員、自転車専用道路の幅員、植樹帯の幅員、歩行者専用道の幅員、歩行者自転車道の幅員
を引数として受け取ることで歩道部の構成を決定している。
また、引数の情報をもとに歩行者が自由に動くことのできる空間を設定することで
歩行者が2次元平面上を移動できるようになり、すれ違いなどの挙動を可能にしている。

\subsection{交差点エージェント}
交差点エージェントはGAMAが提供しているintersectionスキルをもとに作成を行った。
本エージェントは道路の端点を決定するエージェントである。
今回の実験では交差点部ではなく単路部を対象にした実験であるため、
信号処理の実装は行わない。


\section{自転車エージェントのパラメータ設定}
ソーシャルフォースモデルをもとに、自転車エージェントの作成を行うにあたって、
使用するパラメータを選定する必要がある。ソーシャルフォースエージェントの作成に
必要なパラメータは下図の4つである。
そのうち、$\tau$と$\lambda$は
自転車エージェント単独で決定する値である。
A,Bの値は自転車と他のエージェントとの相互作用によって
用いる値であり、相手のエージェントの種類に依存する値である。
今回の実験では、対自転車、対歩行者、対自動車、対道路の境界の
4種類のパラメータを用意する。
本来は、交通シミュレーションを行う対象の道路にて動画を撮影し、
自転車の軌跡を記録する。
その軌跡と誤差がなるべく生じないようパラメータを
設定するという手法が多く用いられているが、今回の実験では
時間や機材が不足しているためその手順を踏むことができなかった。
そこで、今回の実験では他の似たような実験環境で使用されている
論文のパラメータをそのまま採用する。
本論文では、Quらの研究の値を使用することにした[9]。
この論文では、中国鄭州市の単路部における
電動自転車–自動車混合交通を対象にビデオ撮影による実軌跡データを収集し、
モデルの軌跡との誤差が最小限になるように最尤推定法を用いて
ソーシャルフォースモデルのパラメータを推定している。
ただし、この研究では自転車が歩行者を回避する際に使用するパラメータを
もとめていない。
そこで、自転車が歩行者を回避する際に用いるパラメータについては
Helbing が提唱したソーシャルフォースモデルの原論文を参照する[10]。
この研究では,歩行者同士が相互に回避行動をとる際に用いられる
パラメータが示されている。
本研究では、平地単路における歩行者回避という条件が近いため、
これらの値を初期パラメータとして採用した。


\begin{table}[H]
\centering
\begin{tabular}{|c|c|}
\hline
$\tau$&9 \\ \hline
$\lambda$&0.3 \\ \hline
A\_pedestrian&2 \\ \hline
B\_pedestrian&0.5 \\ \hline
\end{tabular}
\caption{対歩行者に使用するパラメータ}
\label{tab:one_col_six_rows}
\end{table}


\begin{table}[H]
\centering
\begin{tabular}{|c|c|}
\hline
$\tau$&9 \\ \hline
$\lambda$&0.3 \\ \hline
A\_bike&0.42 \\ \hline
B\_bike&8.04 \\ \hline
\end{tabular}
\caption{対自転車に使用するパラメータ}
\label{tab:one_col_six_rows}
\end{table}




\begin{table}[H]
\centering
\begin{tabular}{|c|c|}
\hline
$\tau$&9 \\ \hline
$\lambda$&0.3 \\ \hline
A\_car&8 \\ \hline
B\_car&12 \\ \hline
\end{tabular}
\caption{対自動車に使用するパラメータ}
\label{tab:one_col_six_rows}
\end{table}

% 自転車が道路の境界から受ける斥力を求めるために使用するパラメータは

\begin{table}[H]
\centering
\begin{tabular}{|c|c|}
\hline
$\tau$&9 \\ \hline
$\lambda$&0.3 \\ \hline
A\_road&2.3\\ \hline
B\_road&5\\ \hline
\end{tabular}
\caption{道路の斥力に使用するパラメータ}
\label{tab:one_col_six_rows}
\end{table}

\chapter{実験}
\section{実験目的}
本論文の実験では,道路の拡幅や車線削減によって自転車道を新設することが困難な道路を対象とし,
そのような道路において実施可能な再構築手法が
自転車の安全性および快適性に
どのような影響を及ぼすのかについて交通シミュレーションを用いて
定量的に評価することを目的とする。
また、自転車が自転車歩行者道を走行する条件では
歩行者の安全性への影響についても併せて評価する。

\section{実験環境}
本論文では幅員16mの第三種三級道路を対象とする。
第四種道路を使用せず
第四種道路ではなく第三種道路を選定した理由は
都市部の道路や高速自動車道、自動車専用道路に多い第四種道路では
交通量が多く、その結果として片側一車線道路が少ないためである。
一方、地方部に多く存在する第三種道路では交通量が比較的少ないため
片側一車線道路が広く分布しており、本論文の対象として適していると考えられる。
また、第三種道路の中でも交通量が多い道路ほど事故発生のリスクが高く、
再構築による安全性向上の必要性が大きいと考えられることから
本論文では第三種第二級道路を対象とした。
第三種第一級を採用しない理由としては、道路構造令第5条により車線数が4以上であることが
基本とされているためである。
幅員を16 mとした理由は、安全で快適な自転車利用環境創出ガイドラインにおいて、
片側一車線道路の代表的な例として16m幅員の道路が示されているためであり、
本論文ではこの条件を実験環境として採用する。
なお交差点部は信号制御や交差交通流などネットワーク構造の影響を強く受けるため、
本論文ではそれらの影響を排除し道路構造および再配分方法そのものの
効果を明確に評価する目的から単路部を実験対象とした。
先述した通り、再構築前の道路として安全で快適な自転車利用環境創出ガイドラインに
示されていた道路をそのまま利用する。

\section{交通主体の共通設定}
本研究で用いる歩行者,自転車,自動車の生成条件および基本的な走行ルールは,
特に断りのない限りすべての実験設定において共通とする。

\subsection{歩行者}
政府資料において1日500人以上の歩行者交通がある道路は
交通量が多いとされているため、本研究でもこの基準を採用する。
本研究ではこの条件を満たすように、
歩行者の平均到着間隔が約172秒となる確率過程としてモデル化した。
具体的には歩道の両端点それぞれにおいて,
1/344の確率で歩行者を生成し、
生成位置は歩道内のランダムな位置とすることで
歩行者の到着間隔および空間分布にばらつきを持たせている。
歩行者は基本的に他の歩行者及び道路境界からのみ斥力を受けるものとする。
また、初速は

\subsection{自転車}
政府資料において1日500台以上の自転車交通がある道路は
交通量が多い道路とみなされているため、
本研究でもこの基準を採用する。
自転車は歩行者と同様に,
自転車道の始点において毎秒1/172の確率で生成することで
到着間隔および空間分布にばらつきを持たせている。
なお,自転車歩行者道を対象とする実験設定では,
自転車の双方向通行を許容するため,
道路の両端点において毎秒1/344の確率で生成する。
特に断りのない限り自転車は一方通行とし、
各実験設定で定められた走行空間内を走行する。
また、初速は

\subsection{自動車}
第三種第二級道路における地方部国道の定義として,
1日当たりの交通量が20000台以下であることを踏まえ
本研究では20000台/日を上限として設定する。
自動車は等間隔で走行するため、道路の端点から
4.32秒に1台生成する。



\section{実験設定}
本論文では以下に示す再構築前の道路と5つの道路設計で
再構築後の道路を交通シミュレーションにより比較する。
なお、すべての再構築後の道路設計で道路構造令を遵守している。

\subsection{実験設定1 再構築前の道路}
再構築前の道路として幅員が16mで長さ100mの直線道路を使用する。
本論文では安全で快適な自転車利用環境創出ガイドラインの例を採用する。
規制速度は60km/hとし、ガイドラインで示されている設計速度の数値を使用した。
なお、停車は行うことができない区域を対象とする。
本設定では自転車の横幅に対して路肩幅員が十分でないため、
自転車の生成位置は路肩の中央とする。
また、路肩と車道の間に物理的な分離構造が存在しないことから
自転車は路肩境界ではなく、
車道全体の境界から斥力を受けるものとする。


\begin{figure}[htbp]
  \centering
  \includegraphics[width=15cm]{./設定1.png}
  \caption{実験設定1,2の道路}
  \label{fig:speed}
\end{figure}


\subsection{実験設定2 車道に速度制限をかける}
実験設定2では車道の速度制限を60km/hから40km/hに変更する。
ガイドラインにて示されている交通状況を踏まえた整備形態の選定(完成形態)の考え方 
にて、規制速度の強化や交通量の減少が計画できる場合は車道混在のままで完成形態
とすることができる。
この40km/hという速度は第三種第二級道路の設計速度のうち、
選択しうる中で最低の速度である。
交通量の減少を行うためには迂回路を作成するなどの手段を構築することが考えられるが
ミクロシミュレーションの域を超えているため本論文では行わない。
そのほか、道路の構成や自転車と歩行者の挙動も再構築前の道路と統一する。



\subsection{実験設定3 自転車専用通行帯の設置}
実験設定3では自転車専用通行帯を設置することで、
自転車が車道の路肩を走行する必要のない道路構成とする。
そのためセンターラインの省略を行い自転車専用通行帯の幅員を確保する。
さらに道路構造令第8条第7項に基づいて路肩を省略することで
その幅員を縮小し、車道内に幅員1.5mの自転車専用通行帯を設置する。
本設定においても自転車の横幅に対して自転車専用通行帯幅員が十分でないため、
自転車の生成位置は自転車専用通行帯の中央とする。
また、自転車は自転車専用通行帯の境界線から斥力を受ける。
さらに、自転車と自動車の間に物理的な分離構造が存在しないため、
自転車は自動車からの斥力を受ける。


\begin{figure}[htbp]
  \centering
  \includegraphics[width=15cm]{./setting3.png}
  \caption{実験設定3の道路}
  \label{fig:speed}
\end{figure}


\subsection{実験設定4 自転車専用道路の設置}
実験設定4では自転車専用道路の設置を行い、自転車と車道を物理的に区分する。
実験設定3と同様にセンターラインの省略、路肩の省略を行い自転車専用道路の幅員を確保する。
なお、やむを得ない場合は自転車専用道路の幅員は1.5mに削減することができるとあり、
これを適用する。また、植樹帯を1.5mから1mに削減し、余った空間に車道と自転車専用道路を隔てる
0.5mのガードレールを設ける。この整備により車道、自転車専用道路、および歩道は
相互に物理的に分断される。
本設定ではこの物理的分離により、歩行者および自転車は
他の種類のエージェントから斥力を受けないものとする。
また、横幅に対して幅員が狭いため自転車は自転車専用道路の中央に生成され、
自転車専用道路の境界から斥力を受ける。

\begin{figure}[htbp]
  \centering
  \includegraphics[width=15cm]{./設定4.png}
  \caption{実験設定4の道路}
  \label{fig:speed}
\end{figure}

\subsection{実験設定5 自転車歩行者道に変更}
実験設定5では歩道を自転車歩行者道に変更し、拡張する。
センターラインの省略、路肩の省略、植樹帯の1.5mから1mへの削減を行い自転車歩行者道の幅員を確保する。
自転車歩行者道での対応という資料にのっとり、
自転車歩行者道の幅員を4m確保し、歩道上で自転車の双方向通行を許容する。
本設定では自転車の横幅に対して十分な幅員が設けられているため、
両端から0.5mの位置を除いたランダムな位置に生成される。
また、自転車と歩行者が物理的に分離されていないため、
自転車および歩行者は相互に斥力を受けるものとする。


\begin{figure}[htbp]
  \centering
  \includegraphics[width=15cm]{./設定5.png}
  \caption{実験設定5の道路}
  \label{fig:speed}
\end{figure}

\section{評価指標}
本実験では,道路再構築が自転車に与える影響を
安全性および快適性の観点から定量的に評価するため,
それぞれに対応した評価指標を設定する。
安全性の評価指標として,最近接距離を用いる。
最近接距離とはあるエージェントが道路区間を通過する過程において
他のエージェントと最も接近した際の距離を指す。
この指標を用いることで実際に事故として顕在化しない場合であっても、
潜在的に接触リスクが最も高まった瞬間を捉えることが可能となる。
そのため、事故件数のように長期間の観測を必要とせず、
交通シミュレーションにおいて安全性を評価する指標として有効であると考えられる。
本実験では、自転車については自動車と
物理的に分断されていない実験設定に限り
自動車との最近接距離を測定する。
また、歩行者については,
自転車と物理的に分断されていない実験設定5に限り
自転車との最近接距離を測定する。
なお、エージェントが生成されてから道路区間を通過するまでの間に
対象となる他エージェントとの接近が発生しなかった場合には
最近接距離を定義できないため、
当該サンプルは評価データから除外する。
快適性の評価指標としては平均速度と通過時間を用いる。
平均速度は走行の円滑性や移動の効率性を表す基本的な指標であり、
自転車がどの程度ストレスなく走行できているかを
間接的に評価することができる。
通過時間は生成されてから道路を通過するまでの時間の平均をとる。
道路再構築によって通行空間が変化した際の
走行環境の改善あるいは悪化を定量的に把握するために、
平均速度と通過時間は適切な指標であると判断した。
各々の値の具体的な測定方法として、最近接距離はそれぞれのエージェントが
生成されてから0.1秒ごとに
他のエージェントとの距離を測定し、その値が最近接距離より小さかった場合に最近接距離の値を
更新する。エージェントが道路区間を通過し終えた時点でその最近接距離を記録し、
すべてのサンプルについて平均値を算出する。
通過時間は各エージェントについて目的地に到達した時刻から
生成時刻を差し引くことで求め、得られた通過時間を全サンプルについて平均する。
なお,本研究では評価に用いるサンプル数を十分に確保するため
シミュレーション開始時刻から1時間後までに取得したデータを
評価対象とする。
一定時間にわたって計測を行うことで瞬間的な挙動に左右されにくく、
統計的に安定した評価結果を得ることを目的としている。


\section{実験結果}
本実験を行った結果、自転車と自動車の最近接距離は図4.1のような結果となった。
実験1と2では、自転車と自動車の距離に0.04mしか変わらず、実験1と3では0.32mと
わずかに差が生まれた。


\begin{figure}[htbp]
  \centering
  \includegraphics[width=15cm]{自転車と自動車の距離.png}
  \caption{自転車自動車間の最近接距離の平均}
  \label{fig:bicycle_speed}
\end{figure}



また、自転車と自転車の距離は図4.2のような結果になった。実験4が自転車同士の距離が最も短くなり、
実験5が最も自転車同士の距離が取れているという結果になった。
また、実験1,2,3の中では実験1の0.2mほど距離が長くなっている。

\begin{figure}[htbp]
  \centering
  \includegraphics[width=15cm]{自転車の距離.png}
  \caption{自転車間の最近接距離の平均}
  \label{fig:bicycle_speed}
\end{figure}


実験5で計測した自転車と歩行者の距離は0.48mであった。


%平均速度
自転車の平均速度は下図のような結果になった。
実験1と2と3では自転車の速度にほとんど変化が見られないが、
実験1,2,3と実験4,5では速度に大きな差がみられる。
また、実験4と5では0.5km/hほどの差が生まれた。

\begin{figure}[htbp]
  \centering
  \includegraphics[width=15cm]{自転車速度.png}
  \caption{自転車の平均速度}
  \label{fig:bicycle_speed}
\end{figure}


自転車の通過時間は下図のような結果になった。
実験1、2、3では、通過時間にほとんど差が見られなかった。
また、実験1,2,3と実験4,5では通過時間に大きな差がみられる。
その中でも実験4と5では実験4の方が通過時間が短いことが分かった。


\begin{figure}[htbp]
  \centering
  \includegraphics[width=15cm]{通過時間.png}
  \caption{自転車の通過時間}
  \label{fig:bicycle_speed}
\end{figure}

\chapter{考察}
自転車と自動車の距離の結果から、自転車は路肩ではなく自転車専用通行帯を
走行することで距離を確保することができることが分かった。
また、自動車の制限速度を減少しても確保することができる距離に大きな差がないことが分かった。
次に、自転車同士の距離の測定結果から、自転車専用道を走行したときが自転車同士の
接触リスクが高くなった。
この結果が起こった要因として、自動車を回避する動きが
無くなり、速度が出やすいことが考えられる。
実験2と実験3で大きな違いがみられないことから
幅員が狭いことによる影響は考えにくい。
これらの結果から、自転車同士の安全性を最も確保することができるのは、
車道とも物理的に分離されており自転車同士距離も最も確保することができる
実験設定5の自転車歩行者道を設けるケースであることが分かった。
次に快適性という観点で見ると実験4,5が速度や通過速度の指標が最もよかった。
実験4と5では速度はほとんど変わらないが通過時間は実験4の方が若干短くなっている。
この要因として、実験5では歩行者を回避する動きがあったため走行距離に差が出たものと
考えられる。また、実験1,2,3では自転車の速度に変化が見られないが、通過時間は
実験3が実験1,2に比べ僅かに短くなっている。その理由も先ほどと同様に実験1,2は車両を
よける動きが実験3に比べて大きく働いたためだと考えられる。
今回の結果から、3つの観点からそれぞれの実験の順位を作成した。
\begin{table}[H]
\centering
\begin{tabular}{|c|c|}
\hline
実験1&2 \\ \hline
実験2&3 \\ \hline
実験3&4 \\ \hline
実験4&5 \\ \hline
実験5&1 \\ \hline
\end{tabular}
\caption{自転車同士の距離}
\label{tab:one_col_six_rows}
\end{table}

\begin{table}[H]
\centering
\begin{tabular}{|c|c|}
\hline
実験1&4 \\ \hline
実験2&5 \\ \hline
実験3&3 \\ \hline
実験4&1 \\ \hline
実験5&1 \\ \hline
\end{tabular}
\caption{自転車と自動車の距離}
\label{tab:one_col_six_rows}
\end{table}

\begin{table}[H]
\centering
\begin{tabular}{|c|c|}
\hline
実験1&5 \\ \hline
実験2&4 \\ \hline
実験3&3 \\ \hline
実験4&1 \\ \hline
実験5&2 \\ \hline
\end{tabular}
\caption{自転車の快適性}
\label{tab:one_col_six_rows}
\end{table}

この結果から、それぞれの実験設定別に特徴を挙げていく。
実験設定1,2は自転車同士の距離がとりやすい一方、自動車との接触という危険性がある。
また、自転車の快適性という観点でも他の設定に劣っていた。
実験3は自転車同士の距離があまりとれないというデメリットがある。
実験4は自転車と自動車の距離や自転車の快適性という観点で秀でている一方、
自転車同士の距離が詰まりやすいという危険性があることが分かった。
実験5の結果からは自転車同士の距離が開きやすく、自動車との接触リスクもない。
また、自転車の快適性という観点でも秀でていることが分かった。
よって自転車の安全性や快適性という観点から見れば、実験5の自転車歩行者道を設計する
というケースにするべきだと言える。しかし、実験設定5では歩行者と自転車の接触という唯一の懸念がある。
また、実験4は自転車同士の距離が近いという観点以外では秀でているため
自転車の通行量が少ない道路では選択するべきという風に考えられるが、今回のシミュレーションは
すべての自転車が一方通行を行ったときという前提がある。
もしも逆走する自転車があれば、走行速度が速い分事故を起こした際のリスクが大きく
なるし、自転車専用道ではすれ違いが難しいため快適性の悪化が予想される。

今回の結果から条件付きでベストプラクティスを考えると、自転車が必ず一方通行を
守り、自転車の交通量も多くない箇所では実験4の設定が有効であると考えられる。
また、歩行者や自転車の交通量が少なく、歩行者の安全に目を瞑るのなら、実験5の設定
も有効であると考えられる。


\chapter{おわりに}
本研究の実験結果から,道路再構築に関する単一のベストプラクティスを
一概に提示することは困難であることが示された。道路の形状や周辺環境、
天候条件など本シミュレーションでは考慮できていない要因が多数存在するため
すべての道路条件に適用可能な汎用的な交通シミュレーションを構築することには
一定の限界があると考えられる。
それでも本研究を通して、より現実的な挙動に近づけるための方策として
自転車の走行速度や、走行位置、など個人に依存する特性を変数として
使用することでシミュレーションの精度を向上させる余地があると感じた。
これらを今後の課題として取り組むことで、
本シミュレーションの再現性および汎用性の向上が期待される。


% \begin{thebibliography}{10}
% \item 秋山直毅,藤井秀樹,吉村忍.Social Force Modelを用いたパーソナルモビリティと歩行者の混合交通のマルチエージェントシミュレーション.
% 人工知能学会全国大会論文集.2023
% \item 大西祐未,八木雅治,亀岡大悟.再現性の高いミクロ交通シミュレーションを用いた渋滞発生メカニズムの検証. 四国地方整備局管内技術・業務研究発表会論文集.2020.2020.p.4144
% \item 大井 啓史, 江戸 元希, 有村 幹治.社会実験データとミクロ交通シミュレーションを用いた信号現示パターンの最適化,
% 土木学会論文集.2023.79巻7号
% \item 紫藤聖也,沼田仲穂,澤田東一,松井雄馬.車道に於ける自動車と自転車の並走に関する研究.
% 自動車技術会論文集.2013.44巻.1号.p105-110
% \item 横関 俊也1・萩田 賢司2・矢野 伸裕3・森 健二,自転車の通行方法と事故の危険性について-歩道のある単路部での検討-.
% 土木学会論文集.2016.Vol73.No.5.p.I_1095-I_1104
% \item 齊藤 大将,屋井 鉄雄.自転車通行環境を考慮した 道路空間再構築の効果と影響に関する研究.
% 土木学会論文集.2023.Vol78.No.5.pI_671-I_681
% \item 三好平祐,稲田健人,森本章倫.自転車交通を考慮した自動運転社会の道路空間配分に関する研究.
% 交通工学研究発表会論文集.2022.vol42.No.110.p729-736
% \item DirkHelbing,PeterMolnar.Social force model for pedestrian dynamics.
% Physical Review E.1995.vol51.No5.p4282-4286
% \end{thebibliography}

\begin{thebibliography}{99}
\bibitem{akiyama}
秋山直毅,藤井秀樹,吉村忍.
Social Force Modelを用いたパーソナルモビリティと歩行者の混合交通のマルチエージェントシミュレーション.
人工知能学会全国大会論文集.2023
\bibitem{onishi2020}
大西祐未,八木雅治,亀岡大悟,
「再現性の高いミクロ交通シミュレーションを用いた渋滞発生メカニズムの検証」,
四国地方整備局管内技術・業務研究発表会論文集,2020,pp.~41--44.

\bibitem{ooi2023}
大井啓史,江戸元希,有村幹治,
「社会実験データとミクロ交通シミュレーションを用いた信号現示パターンの最適化」,
土木学会論文集,Vol.~79,No.~7,2023.

\bibitem{shido2013}
紫藤聖也,沼田仲穂,澤田東一,松井雄馬,
「車道における自動車と自転車の並走に関する研究」,
自動車技術会論文集,Vol.~44,No.~1,2013,pp.~105--110.

\bibitem{yokozeki2016}
横関俊也,萩田賢司,矢野伸裕,森健二,
「自転車の通行方法と事故の危険性について――歩道のある単路部での検討――」,
土木学会論文集,Vol.~73,No.~5,2016,pp.~I\_1095--I\_1104.

\bibitem{saito2023}
齊藤大将,屋井鉄雄,
「自転車通行環境を考慮した道路空間再構築の効果と影響に関する研究」,
土木学会論文集,Vol.~78,No.~5,2023,pp.~I\_671--I\_681.

\bibitem{miyoshi2022}
三好平祐,稲田健人,森本章倫,
「自転車交通を考慮した自動運転社会の道路空間配分に関する研究」,
交通工学研究発表会論文集,Vol.~42,No.~110,2022,pp.~729--736.

\bibitem{helbing1995}
D.~Helbing and P.~Moln\'ar,
``Social force model for pedestrian dynamics,''
\textit{Physical Review E}, Vol.~51, No.~5, pp.~4282--4286, 1995.

\bibitem{sfmp}
Zhao-wei Qu, Ning-bo Cao, Yong-heng Chen, Li-ying Zhao, Qiao-wen Bai, and Rui-qi Luo,「Modeling electric bike–car mixed flow via social force model」
Advances in Mechanical Engineering,Vol.9,No9,2017,pp.~1--14.
\item https://www.mlit.go.jp/road/sign/kijyun/pdf/20190705tuukoutai.pdf
\end{thebibliography}

\end{document}