\documentclass[a4paper, 11pt]{jsreport}
\usepackage{amsmath}
\usepackage{amssymb}
\usepackage[dvipdfmx]{graphicx}
\usepackage{enumitem}
\usepackage[nobreak]{cite}
\usepackage{float}


\begin{document}

\title{\LaTeX 卒論}
\author{横田祐汰}
\date{\today}
\maketitle

% \begin{abstract}

% \end{abstract}

% \tableofcontents

\chapter{はじめに}
\section{研究背景}
% 昨今環境への配慮や健康増進の観点から、移動手段としての自転車利用が推進されている。
% 一方で法制度の改正により、自転車は原則として車道を通行する車両と
% 位置付けられており、
% 令和8年からは歩道での走行が処罰の対象となる。
% このような背景から道路には自転車が安全に走行できる機能を備えることが求められている。
% 国土交通省は安全で快適な自転車利用環境創出ガイドラインを策定し、
% 新設道路における自転車通行空間の整備や既存道路の再整備を推進している。
% しかしながら、現状では自転車通行空間の整備は十分に進んでいない。
% 国土交通省が公開している「安全で快適な自転車利用環境創出ガイドライン」
% の概要・改定のポイントについてという資料内で
% 自転車道や自転車専用通行帯の延長は微増

% その要因として専用自転車道、自転車通行帯、歩行者自転車道など複数の整備手法の中から、
% どの方式を選択すべきかが必ずしも明確でない点が挙げられる。
% また、各整備手法を選択した場合に生じる利点や課題について、
% 定量的・体系的に示した研究が十分でないことも課題である。
% さらに、現実的には道路の新規拡幅は困難である場合が多く、
% 既存の道路空間を再配分する形での整備が選択されることが多い。
% したがって、限られた道路幅員の中で
% どのような自転車通行空間の整備が有効であるかを検討することが重要である。
昨今、環境への配慮や健康増進の観点から移動手段としての
自転車利用が推進されている。
政府は平成29年度に自転車活用推進法を施行して以来、
自転車利用の促進を図るとともに自転車・歩行者・自動車が安全に通行できるよう
道路環境整備に関する指針を示してきた。
しかし、これまでに実施された道路環境整備の多くは既存の車道の路肩に線を引いて
自転車の通行位置を示すにとどまる車道混在型の整備であった。

その背景として政府は自転車道の設置に必要とされる幅員(2m以上)を確保できない道路が多く、
専用自転車道の整備が困難な状況が多数生じている点を挙げている。
そこで政府は令和元年に道路構造令を改正し、
自転車通行帯という新たな整備手法を規定することで
限られた道路空間においても歩行者の安全を確保しながら
道路環境整備を行うことのできる選択肢を提示した。
しかし、令和6年に政府が公表したデータによると整備された道路約7500kmのうち、9割以上にも
のぼる約6600kmが車道混在型の整備手法を採用している。
このような状況が生じている要因として道路拡幅が容易に行えないという物理的制約に加え、
専用自転車道、自転車通行帯、歩行者自転車道といった複数の整備手法の中から、
どの方式を選択すべきかが必ずしも明確でない点が挙げられる。
また、それぞれの整備手法を選択した場合に生じる利点や課題について検討した研究が少ないため、
費用や時間を要しない車道混在型の整備が暫定的な措置として選択されている可能性も考えられる。
\section{研究目的}
本論文の目的は地方部に多く存在する第三種道路の片側一車線道路を対象として、
道路拡幅を伴わない再配分による自転車が安全に走行するための整備方法を
交通シミュレーションを使用して多角的に比較・評価することである。

再構築を行うことで生じる自転車・自動車・歩行者の安全性や快適性への影響を定量的に分析し、
安全性や走行速度など、整備目的に応じた適切な道路設計手法の選択に資する知見を得ることを目指す。

\chapter{関連研究}

\section{交通シミュレーションに関する研究}
これまで自転車モデルを使用して交通流を
シミュレーションした研究は数多く行われている。
文献[1]
%https://www.jstage.jst.go.jp/article/pjsai/JSAI2023/0/JSAI2023_1F3GS505/_pdf/-char/ja 
ではソーシャルフォースモデルを用いて歩行者と電動スクーター、セグウェイ
のモデルを構築して混合交通シミュレーションを行った。
この論文では低密度と高密度それぞれについて歩行者流を
作り、低密度の歩行者流ではパーソナルモビリティが歩行者を追い越しながら移動する
様子が観測された一方、高密度の歩行者流では歩行者を追い越すことが困難になるなど
現実に即した挙動が観測されている。

より実践的に交通シミュレーションを活用した研究として次の論文が挙げられる。
文献[2]
% https://www.skr.mlit.go.jp/kikaku/kenkyu/r2/ronbun/I-41.pdf
では、香川県丸亀市の実在する道路を対象として
交通シミュレーションにより交通状況を再現し、
渋滞発生の要因となり得る因子の調査を行っている。
文献[3]
% https://www.jstage.jst.go.jp/article/jscejj/79/7/79_22-00200/_pdf/-char/ja
では交通状況ついて実データを集計し、それをもとに
ミクロ交通シミュレーションを作成することで、赤信号・青信号それぞれの
オフセット時間の最適なパターンについて調査を行っている。

これらの研究のように、マルチエージェントシステム(以下、MASとする)を用いて
交通シミュレーションを行った研究は多く行われており、現実世界で起こっている
交通問題の解決に広く役立てられている。

\section{自転車走行空間整備に関する研究}
自転車が路肩を走行する危険性について扱った研究も多く行われている。
文献[4]
% https://www.jstage.jst.go.jp/article/jsaeronbun/44/1/44_20134034/_pdf/-char/en
では自転車が車道を走る車道混在型の危険性について心理的な観点から調査を行った。
まず、自転車利用者に対してアンケートを行い
車道の走行についてのどのように感じているか調査を行った。
次に、ドライブシミュレーターを用いて路肩を走行している自転車の追い越しを
車道幅員を変化させながら被験者に行ってもらい
どの程度危険に感じたか評価してもらった。
2つの実験の結果、自転車は車道の走行に危険意識を感じており、
自動車の運転手は速度が速いほど、自転車との距離が短いほど
危険に感じることが分かった。
文献[5]
% https://www.jstage.jst.go.jp/article/jscejipm/72/5/72_I_1095/_pdf/-char/ja
では千葉県東葛地域の交通事故統計データを用いて歩道走行と比較した車道走行(左側通行)の危険性
について検証している。
この検証の結果車道走行の方が危険性が高いことが判明しており、筆者も自転車専用帯の
設置を提言している。

これまでの研究の結果から、自転車が車道の路肩を走行するのは
心理的にも、客観的な事故件数からも危険であり自転車道の整備
の必要性がわかる。

\section{道路の再構築・再配分に関する研究}
ミクロな道路空間の再構築を検討した研究として
文献[6]
% https://www.jstage.jst.go.jp/article/jscejipm/78/5/78_I_671/_pdf/-char/ja
が挙げられる。
本研究では、両側6車線を有する国道1号線の一部区間を対象とし、
自転車通行空間の整備に関する検討を行っている。
対象区間は、自転車および歩行者の交通量が多いにもかかわらず、
安全に共存できる走行環境が十分に整備されていないという課題を抱えている。
そこで、道路空間の再配分による自転車通行環境の改善を目的として、
交通シミュレーションを用いた比較検討が行われた。
単路部では、車線を1車線削減し、自転車専用通行帯を設ける再構築案を設定し、
現況との比較を行っている。
また、交差点部においては、現況の車道混在型と車線再配分を伴い自転車専用通行帯を導入した案の2種類を対象として、
交通シミュレーションにより評価を行っている。
評価指標としては、所要時間、旅行速度に加え、SSAMを用いた代理安全指標(衝突危険事象数)を用い、
再構築による交通影響および安全性の変化を比較している。

文献[7]
% http://library.jsce.or.jp/jsce/open/00039/201105_no43/pdf/388.pdf
では、自動車の交通量が多いため自転車の約98%が歩道を走行している
道路を対象にして改善するべく自転車道の4つの案を区間ごとに道路に施し
交通実態調査、アンケート調査、街頭インタビューを行うことで
どの案が総合的な評価が高くなるか調査している。

道路の再構築を行う上で既存の幅員から拡幅などを行わず道路の再配分のみを
行う前提としてどのような適切な道路の在り方について調査した研究も存在する。
文献[8]
% https://www.jstage.jst.go.jp/article/jsteproceeding/42/0/42_729/_pdf/-char/ja
では、道路全体の幅員を固定させ、道路構造令にのっとったカーブサイドの整備方法について
停車量や交通量を変化させながら交通シミュレーションを用いて
比較検討を行い、交通条件ごとのカーブサイドの最適な整備方法の提言を行った。

これらの研究から道路の再構築を行う方法として実データを用いて
MASを用いたミクロな交通シミュレーションを
行うという方法や道路の一部を一時的に再編することでどの再構築方法が
客観的な指標や住民の心理的な評価が高いかを調査することで道路の
編成方法を決定するという手法があることが分かった。





\section{本研究の位置づけと新規性}
これまでの研究から自転車が車道を走行する危険性や、
MASを用いた交通シミュレーションが道路の再構築
方法を検討するうえで有効であることなどが分かった。
一方で自転車モデルを用いてシミュレーションを行い、
道路の再構築のベストプラクティスを検討した研究は少ない。
その上、ほとんどが都市部を対象とした片側複数車線の道路を
扱った研究である。
また、地方部は都市部と比較して道路の再構築を行う上での
予算も少ないため検証のために実験を行う余地も少ない
と考えられる。
そこで本研究ではこれまで扱われてこなかった再拡幅のできない第三種・片側一車線道路を
対象とすることで地方で道路の再整備を行う際の一つの判断材料と
なることを目指す。


\chapter{モデルの概要}
\section{シミュレーション全体構成}
本論文では、GAMAをプラットフォームとして使用して実験環境を構築する。
GAMAは定義や挙動の記述を行うことで
エージェントのモデルを作成することができる。
そのモデルをmainファイルで呼び出して
インスタンスの作成を行うことで、
エージェント同士が有機的に作用しあうマルチエージェントシステムを
作成することができる。
本論文の実験では車道モデル、歩道モデル、交差点モデル、歩行者モデル、自転車モデル、自動車モデルの6つを
交通シミュレーションに使用する。自転車モデルはソーシャルフォースモデルをもとに作成し、
道路モデル、歩行者モデル、自動車モデルはGAMA上の既存のモデルを改良し実験に利用する。
本実験では以下のステップで行う。
\begin{enumerate}
  \item 初期化
  \item シミュレーション
  \item 計測
\end{enumerate}



%本実験の環境の説明
\newpage
%図の挿入
\newpage
%説明の続き



\section{ソーシャルフォースモデルの概要}
ソーシャルフォースモデルはDirkHelbingとP'eter Moln'arが文献[9]
% https://arxiv.org/pdf/cond-mat/9805244
で提唱したモデルで、交通シミュレーションなど回避を伴うシミュレーションを
行う際に利用されることが多い。これはエージェントを目標地点へ向かう引力、
他のエージェントから受ける斥力、道路の境界から受ける斥力に分解して個別に計算した
結果を足し合わせることでエージェントの行動を再現するモデルである。
このモデルを改良して歩行者モデルを作成し、その後歩行実験を行ってモデルの
精度をQ_K図にて検証した実験では1平方メートル内に4人の歩行者が存在する
密度以下の環境では高い精度で歩行者を再現できていることが分かった。(文献[1])
ソーシャルフォースモデルは複雑な交通流を再現する場合においても
各エージェントに対して行動規則を定義することで
エージェントが自律的に挙動するモデルであり
マルチエージェントシステムの構築に非常に適している。
そのため、本研究ではエージェントモデルの構築に
ソーシャルフォースモデルを採用した。



\section{モデルの構築と概要}
\subsection{自転車モデル}
自転車モデルはソーシャルフォースモデルをもとに本実験に合わせて一部改良を加えて作成した。
ソーシャルフォースモデルの欠点として、双方向からのすれ違いが発生するシミュレーションにおいて
衝突が起きやすいことが挙げられる。その原因として斥力の働くメカニズムに原因がある。
エージェントの斥力は他のエージェントの現在地とΔt秒後の位置の両方から逃げることができるように
二つの位置から自分の位置へのベクトルの平均をとり、その方向を斥力のベクトル方向とする。
しかし、自分の走行しているベクトル上に自分と逆方向に走行しているエージェントがあった場合、
走行方向と180度反対方向に斥力がかかり回避ではなく減速方向に作用するのみに終始してしまい、
結果として他のエージェントとの衝突が起きてしまう。
その事態を回避するべく進行方向と斥力が打ち消しあう方向に働くとき進行方向に
垂直なベクトルを加えることにした。
% 斥力と進行方向の内積が1-10^-2以上の時進行方向ベクトルを0.1倍したベクトルを加える。
尚、かける方向として他のエージェントより左側に位置していた時左方向に、右側に位置していた時
右方向にかかる。
また、同じ位置であった場合、ランダムにかけるものとする。
本モデルは出発点や目的地のほか、ソーシャルフォースモデルの作成に
用いるパラメータを引数として設定することで加速度や他の自転車エージェントや
歩行者エージェント、自動車エージェント、道路から受ける斥力をどの程度強く回避するかという情報を
決定する。

\subsection{自動車モデル}
自動車モデルはGAMAが提供しているdrivingスキルをもとに作成を行った。
本モデルは速度や加速度という走行速度に関する情報と
出発点や目的地や使用する道路という走行場所に関する情報を
を引数に設定することで、走行を実現している。

\subsection{歩行者モデル}
歩行者モデルはGAMAが提供しているpedestrianスキルをもとに作成を行った。
本モデルもは速度と出発点と目的地を
を引数に設定することで、走行を実現している。

\subsection{車道モデル}
車道モデルは車両が通過する箇所の走行や描画処理を行う。
なお、自転車が自転車専用道や自転車歩行者道を走行する際は例外として、歩道モデルが描画処理を行う。
本モデルは道路の中心点、センターラインの幅員、車道の幅員、路肩の幅員、自転車専用通行帯
を引数として受け取ることで車道の構成を決定している。

\subsection{歩道モデル}
歩道モデルは歩行者が通過する箇所の走行や描画処理を行う。
先述した通り自転車が自転車専用道や自転車歩行者道を走行する際は例外として、
歩道モデルで描画した箇所を走行する。
本モデルは道路の中心点、ガードレールの幅員、自転車専用道の幅員、植樹帯の幅員、歩行者専用道の幅員、歩行者自転車道の幅員
を引数として受け取ることで歩道部の構成を決定している。
また、引数の情報をもとに歩行者が自由に動くことのできるスペースを設定することで
歩行者が2次元で動くことができるようになり、すれ違いなどの挙動を可能にしている。

\subsection{交差点モデル}
交差点モデルはGAMAが提供しているintersectionスキルをもとに作成を行った。
本モデルは道路の端点を決定するモデルである。
今回の実験では交差点部ではなく単路部を対象にした実験であるため、
信号処理の実装は行わない。


\section{自転車モデルのパラメータ設定}
ソーシャルフォースモデルをもとに自転車モデルの作成をするにあたって
使用するパラメータを選定する必要がある。
今回の実験では、対自転車、対歩行者、対自動車の3種類のパラメータを使用する。
ソーシャルフォースモデルの作成に必要なパラメータは下図の4つである。
本来は、交通シミュレーションを行う対象の道路にて動画を撮影し、
自転車の軌跡を記録する。
その軌跡と誤差がなるべく生じないようにソーシャルフォースモデルのパラメータを
設定するという手法が多く用いられているが、今回の実験では
時間や機材が不足しているためその手順を踏むことができなかった。
そこで、今回の実験では他の似たような実験環境で使用されている
論文のパラメータをそのまま採用する。

\begin{table}[H]
\centering
\begin{tabular}{|c|}
\hline
$\tau$ \\ \hline
$\lambda$ \\ \hline
A \\ \hline
B \\ \hline
\end{tabular}
\caption{SFMで使用するパラメータ}
\label{tab:one_col_six_rows}
\end{table}

自転車が歩行者を回避するために設定するパラメータは
文献[9]を使用する。
\vspace{10\baselineskip}


 
\begin{table}[H]
\centering
\begin{tabular}{|c|}
\hline
$\tau$ \\ \hline
$\lambda$ \\ \hline
A \\ \hline
B \\ \hline
\end{tabular}
\caption{対歩行者に使用するパラメータ}
\label{tab:one_col_six_rows}
\end{table}

自転車が自転車を回避するために設定するパラメータは
文献[10]
%https://www.sciencedirect.com/science/article/abs/pii/S1569190X20301921
を参考に使用する。
この論文では平地での自転車同士の相互作用を扱っている点から、
本論文でも十分に使用可能であると考えた。


\begin{table}[H]
\centering
\begin{tabular}{|c|}
\hline
$\tau$ \\ \hline
$\lambda$ \\ \hline
A \\ \hline
B \\ \hline
\end{tabular}
\caption{対自転車に使用するパラメータ}
\label{tab:one_col_six_rows}
\end{table}

自転車が車両を回避するために使用するパラメータは
文献[11]
%https://journals.sagepub.com/doi/epub/10.1177/1687814017719641
を使用する。
この論文内では、自転車と車両が相互作用する環境を扱っている。

\begin{table}[H]
\centering
\begin{tabular}{|c|}
\hline
$\tau$ \\ \hline
$\lambda$ \\ \hline
A \\ \hline
B \\ \hline
\end{tabular}
\caption{対車両に使用するパラメータ}
\label{tab:one_col_six_rows}
\end{table}

\chapter{実験}
\section{実験目的}
本論文の実験では,道路の拡幅や車線削減によって自転車道を新設することが困難な道路を対象とし,
そのような道路において実施可能な再構築手法が
自転車,歩行者,および自動車の安全性および快適性に
どのような影響を及ぼすのかを交通シミュレーションにより定量的に評価することを目的とする。

\section{実験環境}
本論文では幅員16mの第三種三級道路を対象とする。
第四種道路を使用せず
第三種道路を選定した理由として道路の車線数は交通量に応じて決定されるため、
都市部の道路や高速自動車道自動車専用道路では
交通量が多く片側一車線道路の割合が低くなる点が挙げられる。
一方、地方部に多く存在する第三種道路では交通量が比較的少ないため、
片側一車線道路が広く分布しており本研究の対象として適していると考えられる。
また、第三種道路の中でも交通量が多い道路ほど事故発生のリスクが高く、
再構築による安全性向上の必要性が大きいと考えられることから
本研究では第三種第二級道路を対象とした。
第三種第一級を採用しない理由としては、道路構造令第5条により、車線数が4以上が基本である
ためである。
幅員を16 mとした理由は、安全で快適な自転車利用環境創出ガイドラインにおいて、
片側一車線道路の代表的な例として16m幅員の道路が示されているためであり、
本研究ではこの条件を実験環境として採用する。
なお、交差点部は信号制御や交差交通流などネットワーク構造の影響を強く受けるため、
本研究ではそれらの影響を排除し道路構造および再配分方法そのものの
効果を明確に評価する目的から単路部を実験対象とした。
先述した通り、再構築前の道路として安全で快適な自転車利用環境創出ガイドラインに
示されていた道路をそのまま利用する。

\section{実験設定}
本論文では以下に示す再構築前の道路と5つの道路設計で
再構築後の道路を交通シミュレーションにより比較する。
なお、すべての再構築後の道路設計で道路構造令を遵守している。
\subsection{実験設定1 再構築前の道路}
再構築前の道路として幅員が16mの直線道路を使用する。
本論文では安全で快適な自転車利用環境創出ガイドラインの例を採用する。
計画交通量は1日当たり4754台とし、これは国土交通省が発表している
令和3年度全国道路・街路交通情勢調査  一般交通量調査
の結果を利用した。
規制速度は60km/hとし、ガイドラインで示されている設計速度の数値を使用する。
歩行者の交通量は
停車は行うことができない地域を対象とする。
%自転車はすべて所定の道路を進むようにする。など細かい注意書き


\subsection{実験設定2 車道に速度制限をかける}
実験設定2では車道の速度制限を60km/hから40km/hに変更する。
ガイドラインにて示されている交通状況を踏まえた整備形態の選定(完成形態)の考え方 
にて、規制速度の強化や交通量の減少が計画できる場合は車道混在のままで完成形態
とすることができる。
この40km/hという速度は第三種第二級道路の設計速度のうち、
選択しうる中で最低の速度である。
交通量の減少を行うためには迂回路を作成するなどの手段を構築することが考えられるが
ミクロシミュレーションの域を超えているため本論文では行わない。
そのほか、道路の構成は再構築前の道路と統一する。

% \vspace{18cm}

\subsection{実験設定3 自転車通行帯の設置}
実験設定3では自転車通行帯を設置し自転車が車道の路肩を走らなくていいように区分を行う。
そのためセンターラインの省略を行い自転車通行帯の幅員を確保する。
また、道路構造令第8条第7項より路肩を設けず、その幅員を縮小する。
そうすることで車道内に幅員1.5mの自転車通行帯を設置する。
本シミュレーションでは、歩行者は車道と植樹帯によって物理的に隔離されているため
自転車や自動車からの影響は受けない。
また、自転車と自動車は自転車通行帯の境界線から斥力を受ける。



\subsection{実験設定4 自転車専用道の設置}
実験設定4では自転車専用道の設置を行い、自転車と車道を物理的に区分する。
実験設定3と同様にセンターラインの省略、路肩の省略を行い自転車専用道の幅員を確保する。
なお、やむを得ない場合は自転車専用道の幅員は1.5mに削減することができるとあり、
これを適用する。
また、植樹帯を1.5mから1mに削減し、余った空間に車道と自転車専用道を隔てる0.5mの物理的な
壁を設ける。
また、歩道と自転車専用道を隔てるように植樹帯を設置する。これを行う根拠として、
自転車は車両に分類されるため、区分でいうと歩行者と自転車、自動車という位置づけになるため。
この整備により車道と自転車専用道と歩道は物理的に遮断されるためそれぞれが他の乗物による
斥力が発生しないものとする。

\subsection{実験設定5 自転車歩行者道に変更}
実験設定5では歩道を自転車歩行者道に変更し、拡張する。
センターラインの省略、路肩の省略、植樹帯の1.5mから1mへの削減を行い自転車歩行者道の幅員を確保する。
自転車歩行者道での対応という資料にのっとり、
自転車歩行者道の幅員を4m確保し、歩道上で自転車の双方向通行を許容する。

\subsection{自転車の走行}
政府が発表している資料では、1日500台以上走行している道路に対して
交通量が多いと言えると目安が示されている。
よって本実験でもこれを採用する。
57/sに一度スポーンさせる。

\subsection{歩行者の走行}
政府が発表している資料では、1日1500台以上走行している道路に対して
交通量が多いと言えると目安が示されている。
よって本実験でもこれを採用する。
57/sに一度スポーンさせる。


\subsection{自動車の走行}
本実験では使用する地方部の平地の国道における
第三種第二級道路の定義が1日当たりの交通量が20000台以下の道路である
ということから上限の20000台を1日に走行させる。
すなわち、4.32秒当たりに一台車をスポーンさせることで本実験ではその環境を作成する。

\section{評価指標}
% \vspace{10\baselineskip}

\section{実験結果}
% \vspace{25\baselineskip}

\chapter{考察}
% \vspace{10\baselineskip}

\chapter{おわりに}
% \vspace{10\baselineskip}


\end{document}