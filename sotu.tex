\documentclass[a4paper, 11pt]{jsreport}
\usepackage{amsmath}
\usepackage{amssymb}
\usepackage[dvipdfmx]{graphicx}
\usepackage{enumitem}
\usepackage[nobreak]{cite}


\begin{document}

\title{\LaTeX 環境構築の手順}
\author{横田祐汰}
\date{\today}
\maketitle
% \begin{概要}

% \end{概要}
\tableofcontents
% \clearpage

\chapter{はじめに}
\section{研究背景}
昨今環境配慮や健康増進の観点から移動手段としての自転車利用が推進されている。
その一方で、近年の法改正により自転車は車道通行が原則とされ、
令和8年からは歩道での走行が処罰の対象となる。
そのため、道路には自転車が安全に走行できる機能を備えることが求められている。
国土交通省は安全で快適な自転車利用環境創出ガイドラインを策定し、
新設道路における自転車通行空間の整備や既存道路の再整備を推進している。
しかしながら、現状では自転車通行空間の整備は十分に進んでいない。
その要因として専用自転車道、自転車通行帯、歩行者自転車道など複数の整備手法の中から、
どの方式を選択すべきかが必ずしも明確でない点が挙げられる。
また、各整備手法を選択した場合に生じる利点や課題について、
定量的・体系的に示した研究が十分でないことも課題である。
さらに、現実的には道路の新規拡幅は困難である場合が多く、
既存の道路空間を再配分する形での整備が選択されることが多い。
したがって、限られた道路幅員の中で
どのような自転車通行空間の整備が有効であるかを検討することが重要である。


\section{研究目的}
本研究の目的は地方部に多く存在する第三種道路の片側一車線道路を対象として、
道路拡幅を伴わない再配分による自転車が安全に走行するための整備方法を
交通シミュレーションにより多角的に比較評価することである。
再構築を行うことで生じる自転車・車両・歩行者の安全性や快適性への影響を定量的に分析し、
安全性や走行速度など、整備目的に応じた適切な道路設計手法の選択に資する知見を得ることを目指す。

\chapter{関連研究}


\section{交通シミュレーションに関する研究}
これまで自転車モデルを使用して交通流を
シミュレーションした研究は数多く行われている。
% 文献1https://www.skr.mlit.go.jp/kikaku/kenkyu/r2/ronbun/I-41.pdf
では香川県丸亀市の実在する道路を対象に交通をシミュレーションにより再現することで
渋滞が起こる原因となりうる因子の調査を行った。
文献2は
これらの研究のように、マルチエージェントシミュレーションを用いた
交通シミュレーションを行った研究は多く行われており、現実世界で起こっている
交通問題の解決に広く役立てられている。


\section{自転車走行空間整備に関する研究}



\section{道路再構築・再配分に関する研究}
道路の再構築を行う上で既存の幅員から拡幅などを行わず道路の再配分のみを
行う前提としてどのような適切な道路の在り方について調査した研究も存在する。
% 文献1https://www.jstage.jst.go.jp/article/jsteproceeding/42/0/42_729/_pdf/-char/ja
では、道路全体の幅員を固定させ、道路構造令にのっとったカーブサイドの整備方法について
停車量や交通量を変化させながら交通シミュレーションを用いて
比較検討を行い、交通条件ごとのカーブサイドの最適な整備方法の提言を行った。
文献2





\section{本研究の位置づけ}



\chapter{モデルの概要}
\section{シミュレーション全体構成}
本論文ではGAMAをミドルウェアとして使用し、実験環境を構築する。
GAMAではモデルを作成することができ、それをmainファイルで設置し
実行することでエージェント同士が有機的に作用しあう。
実験では道路モデル、歩行者モデル、自転車モデル、車両モデルの4つを
交通シミュレーションに使用する。
中でも自転車モデルはソーシャルフォースモデルをもとに作成し、
道路モデル、歩行者モデル、車両モデルはGAMA上の既存のモデルを改良し実験に利用する。

\section{ソーシャルフォースモデルの概要}
ソーシャルフォースモデルはDirk HelbingとP'eter Moln'arが
提唱したモデルで交通シミュレーションなど回避を伴うシミュレーションを
行う際に利用されることが多い。これはエージェントを目標地点へ向かう引力、
他のエージェントから受ける斥力、道路の境界から受ける斥力に分解し個別に計算した
結果を足し合わせることでエージェントの行動を再現するモデルである。
このモデルを改良して歩行者モデルを作成し、その後歩行実験を行ってモデルの
精度をQ_K図にて検証した実験では1平方メートル内に4人の歩行者が存在する
密度以下の環境では高い精度で歩行者を再現できていることが分かった。

\section{自転車モデルの構築と概要}
自転車モデルはソーシャルフォースモデルをもとに本実験に合わせて一部改良を加えて作成した。
ソーシャルフォースモデルの欠点として、双方向からのすれ違いが発生するシミュレーションにおいて
衝突が起きやすいことが挙げられる。その原因として斥力の働くメカニズムに原因がある。
エージェントの斥力は他のエージェントの現在地とΔt秒後の位置の両方から逃げることができるように
二つの位置から自分の位置へのベクトルの平均をとり、その方向を斥力のベクトル方向とする。
しかし、自分の走行しているベクトル上に自分と逆方向に走行しているエージェントがあった場合、
走行方向と180度反対方向に斥力がかかり回避ではなく減速方向に作用するのみに終始してしまい、
結果として他のエージェントとの衝突が起きてしまう。
その事態を回避するべく進行方向と斥力が打ち消しあう方向に働くとき進行方向に
垂直なベクトルを加えることにした。
% 斥力と進行方向の内積が1-10^-2以上の時進行方向ベクトルを0.1倍したベクトルを加える。
尚、かける方向として他のエージェントより左側に位置していた時左方向に、右側に位置していた時
右方向にかかる。
また、同じ位置であった場合、ランダムにかけるものとする。

\section{自転車モデルのパラメータ設定}
ソーシャルフォースモデルをもとに自転車モデルの作成をするにあたって
使用するパラメータを選定する必要がある。
また、対自転車、対歩行者、対自動車の3種類のパラメータを使用する。




\chapter{実験}
\section{実験目的}
本論文の実験では,道路の拡幅や車線削減によって自転車道を新設することが困難な道路を対象とし,
そのような道路において実施可能な再構築手法が
自転車,歩行者,および車両の安全性および快適性に
どのような影響を及ぼすのかを交通シミュレーションにより定量的に評価することを目的とする。

\section{実験環境}
本論文では幅員16mの第三種三級道路を対象とする。
第四種道路を使用せず
第三種道路を選定した理由として道路の車線数は交通量に応じて決定されるため、
都市部の道路や高速自動車道自動車専用道路では
交通量が多く片側一車線道路の割合が低くなる点が挙げられる。
一方、地方部に多く存在する第三種道路では交通量が比較的少ないため、
片側一車線道路が広く分布しており本研究の対象として適していると考えられる。
また、第三種道路の中でも交通量が多い道路ほど事故発生のリスクが高く、
再構築による安全性向上の必要性が大きいと考えられることから
本研究では第三種第二級道路を対象とした。
第三種第一級を採用しない理由としては、道路構造令第5条により、車線数が4以上が基本である
ためである。
幅員を16 mとした理由は、安全で快適な自転車利用環境創出ガイドラインにおいて、
片側一車線道路の代表的な例として16m幅員の道路が示されているためであり、
本研究ではこの条件を実験環境として採用する。
なお、交差点部は信号制御や交差交通流などネットワーク構造の影響を強く受けるため、
本研究ではそれらの影響を排除し道路構造および再配分方法そのものの
効果を明確に評価する目的から単路部を実験対象とした。
先述した通り、再構築前の道路として安全で快適な自転車利用環境創出ガイドラインに
示されていた道路をそのまま利用する。

\section{実験設定}
本論文では以下に示す再構築前の道路と5つの道路設計で
再構築後の道路を交通シミュレーションにより比較する。
なお、すべての再構築後の道路設計で道路構造令を遵守している。
\subsection{実験設定1 再構築前の道路}
再構築前の道路として幅員が16mの直線道路を使用する。
本論文では安全で快適な自転車利用環境創出ガイドラインの例を採用する。
% 具体的には、
計画交通量は1日当たり4754台とし、これは国土交通省が発表している
令和3年度全国道路・街路交通情勢調査  一般交通量調査
の結果を利用した。
規制速度は60km/hとし、ガイドラインで示されている設計速度の数値を使用する。
歩行者の交通量は
停車は行うことができない地域を対象とする。
%自転車はすべて所定の道路を進むようにする。など細かい注意書き


\subsection{実験設定2 車道に速度制限をかける}
実験設定2では車道の速度制限を60km/hから40km/hに変更する。
ガイドラインにて示されている交通状況を踏まえた整備形態の選定(完成形態)の考え方 
にて、規制速度の強化や交通量の減少が計画できる場合は車道混在のままで完成形態
とすることができる。
この40km/hという速度は第三種第二級道路の設計速度のうち、
選択しうる中で最低の速度である。
交通量の減少を行うためには迂回路を作成するなどの手段を構築することが考えられるが
ミクロシミュレーションの域を超えているため本論文では行わない。
そのほか、道路の構成は再構築前と統一する。

\subsection{実験設定3 自転車通行帯の設置}
実験設定3では自転車通行帯を設置し自転車が車道の路肩を走らなくていいように区分を行う。
そのためセンターゼブラの省略を行い自転車通行帯の幅員を確保する。
また、道路構造令第8条第7項より路肩を設けず、その幅員を縮小する。
そうすることで車道内に幅員1.5mの自転車通行帯を設置する。
本シミュレーションでは、歩行者は車道と植樹帯によって物理的に隔離されているため
自転車や車両からの影響は受けない。
また、自転車と車両は自転車通帯の境界線から斥力を受ける。

\subsection{実験設定4 自転車専用道の設置}
実験設定4では自転車専用道の設置を行い、自転車と車道を物理的に区分する。
実験設定3と同様にセンターゼブラの省略、路肩の省略を行い自転車専用道の幅員を確保する。
なお、やむを得ない場合は自転車専用道の幅員は1.5mに削減することができるとあり、
これを適用する。
また、植樹帯を1.5mから1mに削減し、余った空間に車道と自転車専用道を隔てる物理的な
壁を設ける。
また、歩道と自転車専用道を隔てるように植樹帯を設置する。これを行う根拠として、
自転車は車両に分類されるため、区分でいうと歩行者と自転車、車両という位置づけになるため。
この整備により車道と自転車専用道と歩道は物理的に遮断されるためそれぞれが他の乗物による
斥力が発生しないものとする。

\subsection{実験設定5 自転車歩行者道に変更}
実験設定5では歩道を自転車歩行者道に変更し、拡張する。
路肩の省略、植樹帯の1.5mから1mへの削減を行い自転車歩行者道の幅員を確保する。
自転車歩行者道での対応という資料にのっとり、
自転車歩行者道の幅員を4m確保し、歩道上で自転車の双方向通行を許容する。


\section{評価指標}


\section{実験結果}


\chapter{考察}

\chapter{おわりに}

\end{document}

%18ページまで(スペース込み)
%参考文献10個以上