
% CDL thesis style file 2016/01/07
%  based on wuse_thesis.tex
%  
%  2016/01/07 和歌山大学システム工学部デザイン情報学科のスタイルをベース
%  に作成
%
%
\documentclass[11pt]{ujreport}
\usepackage{cdl_thesis}
\usepackage{indentfirst}
%\usepackage{graphicx}  % ←graphicx.styを用いてEPSを取り込む場合有効にする
			% 他のパッケージ・スタイルを使う場合には適宜追加

%%%%%%%%%%%%%%%%%%%%%%%%%%%%%%%%%%%%%%%%%%%%%%%%%%%%%%%%%%%%%%%%%%%%%%%%

%%
%% 主に表紙を作成するための情報
%%

%%  タイトル(修論の場合は英語表記も指定)
\title{卒業・修士論文用スタイルファイルを用いた\\
       \TeX による卒業・修士論文の作成}
\etitle{Test\\Test}

%%  著者名(修論の場合は英語表記も指定)
\author{個等保 手座夫}
\eauthor{Dezao KORABO}

%% 卒業論文・修士論文(以下のどちらかを選択)
\bachelar	% 卒業論文(4年生用)
%\master  	% 修士論文(M2用)

%%  コース・学科
\department{先端社会デザイン}
\edepartment{Computer Science}

%% 卒研3クラス
\class{CE}

%% 回生
\grade{4}

%%  学生番号
\studentid{0000000000-0}

%%  卒業年度(学期)
\gyear{2020}		% 提出年が2016年なら,2015年度

%% 学期
\semester{秋学期}

%%  論文提出日
\date{2020年2月1日}	% 修士の場合は月(2016年2月)までとし,英語表記も指定
%\edate{February 2016}	% 修士の場合,こちら(英語表記)も有効化

%%%%%%%%%%%%%%%%%%%%%%%%%%%%%%%%%%%%%%%%%%%%%%%%%%%%%%%%%%%%%%%%%%%%%%%%

\begin{document}

\maketitle

%%
%%  概要
%%
\begin{abstract}
 本稿では,「立命館大学コラボレーションデザイン研究室卒業論文/修士論文用
 スタイルファイル」を用いて卒業論文/修士論文を作成する方法を解説する.
本稿自身,「立命館大学コラボレーションデザイン研究室卒業論文/修士論文用
 スタイルファイル」を用いて記述されており,例によってその使い方を示して
 いる.

 「立命館大学コラボレーションデザイン研究室卒業論文/修士論文用
 スタイルファイル」では,タイトルページ,概要,目次,参考文献などの書式
 を設定している.

  「立命館大学コラボレーションデザイン研究室卒業論文/修士論文用
 スタイルファイル」
は,
\begin{quote}
  \begin{description}
    \item[\tt cdl\_thesis.sty:] 卒業/修士論文用スタイルファイル
    \item[\tt thesis\_sample.tex:] スタイルファイル利用例
  \end{description}
\end{quote}
からなる.

\end{abstract}

%%  目次
\tableofcontents

%%  図目次 (図目次をいれたければ以下のコメントをはずす)
%\listoffigures
%%  表目次 (表目次をいれたければ以下のコメントをはずす)
%\listoftables

\newpage
\pagenumbering{arabic}	% 以降のページ番号を算用数字に

%%%%%%%%%%%%%%%%%%%%%%%%%%%%%%%%%%%%%%%%%%%%%%%%%%%%%%%%%%%%%%%%%%%%%%%%

%%
%%  本文はここから
%%

\chapter{タイトルページ,概要,目次}

{\bf 「立命館大学コラボレーションデザイン研究室卒業論文/修士論文用スタイ
ルファイル」}では,専用のタイトルページを出力する.
記述すべき項目は,
\begin{itemize}
  \item タイトル
  \item 著者名
  \item 学士(4年)/修士(M2)の設定
  \item 学科名/研究科名
  \item 学生番号
  \item 卒業年度
  \item 論文提出日
\end{itemize}
である.
これらのデータは,\verb|\maketitle|によってタイトルページに出力される.
また,概要の部分において,論文の内容をまとめる.その内容は論文の2ペー
ジ目(タイトルページの次)に出力される.
このソースでは,目次(\verb|\tableofcontents|)を出力している.
他に,図目次(\verb|\listoffigures|),表目次(\verb|\listoftables|)を
出力することもできるので,必要ならばそれぞれのコメントをはずす.
図目次,表目次については,第\ref{chap:fig-tab-exp}章において説明する.

\section{タイトル}
\subsection{title}
論文のタイトルを記述する.

\section{著者}
\subsection{author}
著者名を記述する.

\subsection{bachelar/master}
卒業論文の場合には,\verb|\master|をコメントアウトし,
\verb|\bachelar|を設定する.
修士論文の場合には,\verb|\bachelar|をコメントアウトし,
\verb|\master|を設定する.

\subsection{department}
所属学科を記述する.
当研究室は情報コミュニケーション学科所属なので,``情報コミュニケーショ
ン''とあらかじめ記述している.

\subsection{studentid}
学生番号を記述する.

\subsection{gyear}
卒業年度を記述する.

\section{提出日}
\subsection{date}
論文提出日を記述する.

\chapter{図,表,数式}\label{chap:fig-tab-exp}
論文では,図,表,数式などを効果的に使用する.

\section{図}
{\tt figure}環境を利用することによって図にキャプション
(\verb|\caption|)を付けることができる.図に付けられたキャプションは
\verb|\listoffigures|によって図目次として出力される.図には章ごとに通
し番号が付けられ,キャプションに\verb|\label|を設定しておくと,
``図\ref{fig:sample}''のように\verb|\ref|によって図を番号で参照するこ
とができる.図\ref{fig:sample}に{\tt figure}環境を用いた記述例を示す.

\begin{figure}
  \begin{center}
    ここで図を取り込む.
    % 試しに,tiger.psが自分のマシンのどこに格納されているかを調べて
    % 以下の命令を有効にしてみて下さい.
    % ただし,同時に\begin{document}より前にある\usepackage{graphicx}
    % も有効にする必要があります.
    % 以下の例ではついでに四角で囲っています.
    %\framebox{\includegraphics[width=5cm,clip]{/usr/local/share/ghostscript/7.07/examples/tiger.ps}}
  \end{center}
  \caption{図の例}
  \label{fig:sample}
\end{figure}

また,{\tt graphicx.sty}などのスタイルファイルを利用することによって
EPS形式の図を文章の中に取り込むことができる.
この場合,\verb|\begin{document}|の前に\verb|\usepackage{graphicx}|を
追加する.

\section{表}
{\tt table}環境を利用することによって図と同じように,キャプションをつ
けたり,ラベルにより参照したりすることができる.また
\verb|\listoftables|によって表目次として出力される.
表\ref{tab:sample}に{\tt table}環境で作成した表を示す.

\begin{table}
  \caption{表の例}
  \label{tab:sample}
  \begin{center}
    \begin{tabular}{|c|c|c|}
      \hline
      8 & 3 & 4\\
      \hline
      1 & 5 & 9 \\
      \hline
      6 & 7 & 2 \\
      \hline
    \end{tabular}
  \end{center}
\end{table}

\section{数式}
\TeX では数式のための機能が豊富である.
{\tt equation}環境などを利用することによって数式に番号を付けることがで
きる.図や表と同じくラベルを付けておけば,``式\ref{exp:sample}''のよう
に数式を番号で参照することができる.

\begin{equation}
  y = ax^2 + bx + c \label{exp:sample}
\end{equation}

\chapter{参考文献}

文献を参照する場合には,論文の最後に参考文献として列挙するとともに,
\verb|\cite|を使って,例えば,
\begin{quote}
  文献\cite{latex}によれば…
\end{quote}
や,
\begin{quote}
  …である\cite{latex2e}.
\end{quote}
のように参照する.

文献の列挙には,{\tt thebibliography}環境などを用いる\footnote{使い方
は,この資料のソースを参照.}.

\begin{acknowledgements}
 天のお恵みに感謝いたします.
\end{acknowledgements}

%%
%% 参考文献
%%
\begin{thebibliography}{99}
\bibitem{latex}
    奥村晴彦 著,\LaTeX 入門---美文書作成のポイント---,技術評論社,1993.
\bibitem{latex2e}
    奥村晴彦 著,[改定第3版] \LaTeXe~美文書作成入門,技術評論社,2004.
\bibitem{texbasic}
    OfficeMASA,神代英俊,長島秀行 著,\TeX の基礎,
    ソフトバンクパブリッシング,2002.
\bibitem{latexcomp}
    M. Goossens, F. Mittelbach, A. Samarin 共著,
    アスキー書籍編集部 監訳,The \LaTeX コンパニオン,アスキー出版局,1998.
\end{thebibliography}

\appendix

\chapter{サンプルプログラム}
プログラムリストや実行結果など,本論を補足する上で必要と思われるものがあ
れば付録として付ける.

{
\footnotesize
\begin{verbatim}
#include <stdio.h>
int main(void)
{
    printf("Hello, World!\n");
    return 0;
}
\end{verbatim}
}

\end{document}